\chapter{Fahrzeugmodellierung und Herleitung der kanonischen Zustandsgleichungen}\label{cha:Modellbildung}
Dieses Kapitel dient der Modellierung der Fahrzeugbewegung und der Herleitung der aus dem Fahrzeugmodell resultieren kanonischen Zustandsgleichungen. Zunächst wird dazu die Bewegung des Fahrzeugs in Frenet-Koordinaten relativ zu einer Referenzkurve modelliert. Das Fahrzeug wird dabei als masseloser Punkt angenommen, dessen Bewegung mithilfe eines kinematischen Einspurmodells beschrieben werden kann\footnote{Es wird ausschließlich eine zweidimensionale Bewegung des Fahrzeugs parallel zur Straße berücksichtigt, weshalb die Komponente der Fahrzeughöhe bei der Beschreibung der Fahrzeugposition stets vernachlässigt wird.}. Anschließend wird das Gesamtmodell nach Gleichung \eqref{eq:Sysdyn_z} aus der Modellierung der Fahrzeugdynamik unter Anwendung der Optimalitätsbedingungen \eqref{eq:Zustandsdgl}-\eqref{eq:Steuerungsgleichung} aufgestellt. Schließlich werden verschiedene Modellbeschreibungen mit unterschiedlichem Detailgrad betrachtet und deren Nutzen für die Berücksichtigung der in Kapitel \ref{cha:Komfort} erarbeiteten Komfortmerkmale diskutiert. 
\section{Formulierung der Fahrzeugbewegung in Frenet-Koordinaten}\label{sec:Frenet_KS}
Oftmals ist die Beschreibung der Fahrzeugposition und dessen Bewegung in einem globalen Koordinatensystem nicht relevant. Dies gilt insbesondere bei der Fahrzeugstabilisierung und der Planung optimaler Trajektorien zur Durchführung von bestimmten Fahrmanövern. In solchen Fällen bietet sich eine Beschreibung der Fahrzeugbewegung relativ zu einer gegebenen Referenzkurve $\mathcal{R}$ an \cite{Rathgeber.2016}. In der Realität kann diese Referenzkurve beispielsweise durch Fahrbahnmarkierungen gegeben sein. Die Bewegung eines Punkts $\mathcal{P}_r$ entlang der Referenzkurve kann mithilfe der senkrecht zueinander stehenden Vektoren $\vec{\textrm{n}}(s_r)$ und $\vec{\textrm{t}}(s_r)$ beschrieben werden. Der Tangentenvektor $\vec{\textrm{t}}(s_r)$ ist dabei ausgehend von $\mathcal{P}_r$ immer tangential zu $\mathcal{R}$ ausgerichtet und der Normalenvektor $\vec{\textrm{n}}(s_r)$ steht senkrecht darauf \cite{Rathgeber.2016}. Da die Krümmung der Referenzkurve \kapparefofs\,veränderlich ist, sind die beiden Vektoren als ortsabhängige Größen bzw. in Abhängigkeit der entlang von $\mathcal{R}$ zurückgelegten Wegstrecke $s_r$ definiert \cite{Werling.2011}. Das Frenet-Koordinatensystem mit dem Fußpunkt $\mathcal{P}_r$ und den beiden Richtungsvektoren $\vec{\textrm{n}}$ und $\vec{\textrm{t}}$ wird im Folgenden auch mit $\mathcal{F}_r = \{\mathcal{P}_r, \vec{\textrm{n}}, \vec{\textrm{t}}\}$ bezeichnet. Des Weiteren wird das fahrzeugfeste Koordinatensystem $\mathcal{F}_f = \{\mathcal{P}_f, \vec{x}_f, \vec{y}_f\}$ definiert. Wie bereits bei $\mathcal{F}_r$ handelt es sich dabei um ein rechtwinkliges Koordinatensystem. Dessen Ursprung liegt im Schwerpunkt $\mathcal{P}_f$ des Fahrzeugs, welches als punktförmig angenommen wird. Die Vektoren $\vec{x}_f$ und $\vec{y}_f$ beschreiben die Bewegung aus Sicht des Fahrzeugs, wobei $\vec{x}_f$ entlang der Längsachse des Fahrzeugs nach vorne gerichtet ist und $\vec{y}_f$ entsprechend orthogonal dazu nach links orientiert ist. Wie bereits erläutert ist die Beschreibung der Fahrzeugbewegung in einem globalen Koordinatensystem für die Planung der optimalen Trajektorien nicht von Interesse. Dennoch wird ein globales Koordinaten System $\mathcal{F}_g = \{0, \vec{x}_g, \vec{y}_g\}$ definiert, welches im Ursprung liegt und dessen Richtungsvektoren $\vec{x}_g$ und $\vec{y}_g$ aus Sicht der Vogelperspektive stets nach rechts bzw. nach oben zeigen. Dieses Koordinatensystem wird insbesondere für die Visualisierung der Fahrzeugbewegung verwendet.

Alle Koordinatensysteme und die Fahrzeugbewegung sind in Abbildung \ref{fig:Fahrzeugmodell} schematisch dargestellt. Durch den Pfeil, der auf einigen der Größen dargestellt ist, soll hervorgehoben werden, dass es sich dabei um vektorielle Größen handelt. 
\begin{figure}[h]
	\centering
	\fontsize{24pt}{16pt}\selectfont
	\resizebox{0.8\textwidth}{!}{
		\import{./Bilder/Inkscape/}{Fahrzeug.pdf_tex}
	}
	\caption{Kinematisches Einspurmodell eines Fahrzeugs, das sich entlang einer vorgegebenen Referenzkurve $\mathcal{R}$ bewegt. }
	\label{fig:Fahrzeugmodell}
\end{figure}

\subsection{Beschreibung der Relativbewegung}\label{subsec:Relativbewegung}
Die Referenzkurve $\mathcal{R}$ wird durch die wegabhängige Krümmung \kapparefofs\,beschrieben. Der Winkel \psiref\, stellt dabei den Winkel zwischen $\mathcal{R}$ und dem globalen Koordinatensystem $\mathcal{F}_g$ dar. Es wird vereinfachend angenommen, dass keine Schwimmbewegungen im Fahrzeug auftreten. Das bedeutet, dass der Kurswinkel des Fahrzeugs mit dem Gierwinkel $\psi$ übereinstimmt und die Fahrzeuggeschwindigkeit $v$ immer entlang der Längsachse wirkt, also $\dot{x}_f = v$ gilt. Des Weiteren wird mit $d_r$ der seitliche Abstand zwischen dem Fahrzeug und der Referenzkurve eingeführt, wobei $d_r$ wie bereits $\vec{\textrm{n}}(s_r)$ senkrecht zur Referenzkurve orientiert ist und damit dem normalen Abstand zwischen $\mathcal{P}_f$ und $\mathcal{P}_r$ entspricht \cite{Rathgeber.2016}. Der seitliche Abstand ist dabei so definiert, dass $d_r$ positive Werte annimmt, wenn sich das Fahrzeug links von $\mathcal{R}$ befindet. Der Winkel zwischen der Bewegungsrichtung des Fahrzeugs und der Referenzkurve wird durch $\psi_r$ beschrieben und ist als \begin{equation}
	\psi_r = \psi - \psiref
\end{equation}
definiert \cite{Rathgeber.2016}. Ist der Gierwinkel $\psi$ des Fahrzeugs entgegen dem Uhrzeigersinn bezüglich $\mathcal{F}_g$ größer als der Winkel $\psi_{ref}$ der Referenzkurve, nimmt $\psi_r$ positive Werte an. Mit diesen Definitionen ergeben sich nach \cite{Rathgeber.2016} die folgenden Bewegungsgleichungen:
\begin{align}
	\dot{s}_r &= \frac{v\cos{\psi_r}}{1-d_r\kapparefofs} \label{eq:dsr} \\
	\dot{d}_r &= v\sin{\psi_r} \label{eq:ddr} \\
	\dot{\psi_r} &= \dot{\psi} -\dot{\psi}_\textrm{ref} = \kappa v - \frac{\kapparefofs v\cos{\psi_r}}{1-d_r\kapparefofs} \label{eq:dpsir}
\end{align}
Die Größe $\kappa$ in Gleichung \eqref{eq:dpsir} steht für die tatsächlich vom Fahrzeug durchfahrene Krümmung. Je nach Modellordnung kann diese als Systemzustand oder auch als Stellgröße verwendet werden. Die \gls{DGL} für die Fahrzeuggeschwindigkeit $v$ und die Längsbeschleunigung $a_x$ lauten
\begin{align}
\dot{v} &= a_x \label{eq:dv} \\
\dot{a}_x &= j_x \label{eq:dax}\,,
\end{align}
wobei $j_x$ den Längsruck bezeichnet. Analog zu $\kappa$ gilt für die Längsbeschleunigung, dass diese entweder als Systemzustand oder als Stellgröße verwendet werden kann. Die Wahl der Stellgrößen hängt von der betrachteten Systemordnung ab und hat dabei Einfluss darauf, ob die Größen Quer- und Längsruck im Gütefunktional berücksichtigt werden können, wobei dies in Abschnitt \ref{sec:Systemdynamik} genauer erläutert wird. Die Fahrzeugposition im globalen Koordinatensystem $\mathcal{F}_g$ kann durch Integration der Gleichungen
\begin{align}
\dot{x}_g &= v\cos{\psi} \\
\dot{y}_g &= v\sin{\psi}
\end{align}
berechnet werden.

\subsubsection{Klothoiden}\label{subsubsec:Klothoiden}
Übergänge zwischen Kurven oder Geraden und Kurven im Straßenverkehr sind im Allgemeinen als Klothoiden angelegt, deren Krümmung sich proportional zur Länge der Kurve verhält \cite{ForschungsgesellschaftfurStraenundVerkehrswesen.2008}. Die Verwendung von Klothoiden im Straßenbau hat den Vorteil, dass sprunghafte Lenkänderungen vermieden und dadurch weiche und stetige Lenkbewegungen ermöglicht werden \cite{ForschungsgesellschaftfurStraenundVerkehrswesen.2008}. Dies hat zur Folge, dass sich die Querbeschleunigung kontinuierlich ändert und Ruck behaftete Änderungen vermieden werden \cite{ForschungsgesellschaftfurStraenundVerkehrswesen.2008}. Die Krümmung einer Klothoide lässt sich alllgemein als 
\begin{equation}
	\kapparefofs = \dkapparef s_r + \kapparefzero
\end{equation}
angeben \cite{Rathgeber.2016}. Mit \kapparefzero\,wird dabei die Anfangskrümmung der Klothoide bezeichnet und \dkapparef\,gibt die konstante Änderung der Krümmung mit dem Streckenverlauf an \cite{Rathgeber.2016}. Die Parametrierung von Klothoiden als Trassierungselement im Bau von Autobahnen ist in \cite{ForschungsgesellschaftfurStraenundVerkehrswesen.2008} geregelt. Die Krümmungsänderung \dkapparef\,ist mit 
\begin{equation}
	\dkapparef = \frac{1}{A^2}\,,\quad\textrm{mit}\, \frac{R}{3}\leq A\leq R
\end{equation}
gegeben, wobei $A$ als Klothoidenparameter bezeichnet wird und $R$ den kleinsten Radius der zu durchfahrenden Klothoide angibt \cite{ForschungsgesellschaftfurStraenundVerkehrswesen.2008}. Die zulässigen Minimalradien von Autobahnkurven sind abhängig von der erlaubten Geschwindigkeit auf dem jeweiligen Streckenabschnitt festgelegt. Im Bereich hoher Geschwindigkeiten ($\valunit{80}{\frac{\unit{km}}{\unit{h}}}\leq v\leq \valunit{130}{\frac{\unit{km}}{\unit{h}}}$), die beispielsweise bei langen Kurven auf Autobahnen vorkommen, ergeben sich damit Werte für den minimalen Klothoidenparameter von $90\leq A\leq 300$ \cite{ForschungsgesellschaftfurStraenundVerkehrswesen.2008}. Für den bei Autobahnausfahrten relevanten niedrigeren Geschwindigkeitsbereich ($\valunit{30}{\frac{\unit{km}}{\unit{h}}}\leq v\leq \valunit{80}{\frac{\unit{km}}{\unit{h}}}$), lässt sich der Bereich analog zu $10\leq A\leq 83,33$ bestimmen \cite{ForschungsgesellschaftfurStraenundVerkehrswesen.2008}.
 
\subsubsection{Vereinfachungen und Annahmen}\label{subsubsec:Vereinfachungen}
Die Referenzkurve dient nicht nur der Beschreibung der Fahrzeugbewegung relativ zu dieser, sondern auch als Orientierung für die abzufahrende Strecke. Daher ist in vielen Anwendungsfällen die Annahme berechtigt, dass sich das Fahrzeug in der Nähe der Referenzkurve befindet und eine ähnliche Orientierung bezogen auf $\mathcal{F}_g$ hat. Mit kleinen Winkeln $\psi_r$ gilt $\sin{\psi_r}\approx\psi_r$ und $\cos{\psi_r}\approx 1$. Die Krümmung ist allgemein als Kehrwert des Radius definiert. Geht man davon aus, dass die kleinsten Wenderkreisradien, die mit Pkw durchfahren werden können, in etwa bei \valunit{5}{m} liegen, so beträgt die größtmögliche Krümmung \valunit{0,2}{\frac{\unit{1}}{\unit{m}}} \textbf{Quelle}. Zusammen mit der Annahme, dass neben $\psi_r$ auch $d_r$ kleine Werte annimmt, kann die Vereinfachung $1-d_r\kapparefofs\approx 1$ getroffen werden, mit der schließlich $\dot{s}_r\approx v$ folgt. Damit gibt $s_r$ die tatsächlich vom Fahrzeug zurückgelegte Strecke in sehr guter Näherung wieder. Es ist jedoch zu beachten, dass diese Annahme ihre Gültigkeit verliert, je weiter sich das Fahrzeug von der Referenzkurve entfernt, da $s_r$ dann nur noch die orthogonal auf $\mathcal{R}$ projizierte Strecke angibt. Eine weitere Vereinfachung, die in einigen Fällen Anwendung findet, ist die Annahme, dass die Krümmung der Referenzkurve \kapparefofs\,konstant ist und nicht von $s_r$ abhängt und damit $\kapparefofs=\kapparefzero$ gilt. Dies ist beispielsweise der Fall, wenn eine reine Geradeausfahrt ohne Krümmung oder eine Kreisfahrt mit konstanter Krümmung betrachtet werden soll

\section{Adjungierte \gls{DGL} und dynamisches Gesamtmodell}\label{sec:Systemdynamik}
Nachdem die Systemdynamik in Frenet-Koordinaten und die Beschreibung der Referenzkurve erläutert wurden, wird nun ein Gütefunktional der Form \eqref{eq:J_dyn} eingeführt, das zum Aufstellen der Hamilton-Funktion und zur Herleitung der Gesamtsystemdynamik verwendet wird und dabei die in Kapitel \ref{cha:Komfort} herausgearbeiteten Kriterien zur Bewertung des Fahrkomforts verwendet.

Die als komfortrelevant erachtete Reisezeit kann über die freie Endzeit $t_f$ des Gütefunktionals berücksichtigt werden und wird dabei als Teil der Endkosten \Vofxoftf\,bestraft. Dadurch wird das Optimierungsziel im Sinne einer kürzeren Reisezeit schneller erreicht. Zudem werden die Größen Quer- und Längsbeschleunigung sowie Quer- und Längsruck bestraft, da diese den Fahrkomfort der \GenderPl{Fahrzeuginsass} maßgeblich beeinflussen. Um zu gewährleisten, dass sich das Fahrzeug immer nahe der Referenzkurve befindet, wird zusätzlich die Querabweichung $d_r$ bestraft. Somit ergibt sich das verwendete Gütefunktional zu
\begin{equation}
	J(\xoft,\uoft,t,t_f) = t_f + \int_{t_0}^{t_f} \frac{1}{2}\fax a_x^2 + \frac{1}{2}\fjx j_x^2 + \frac{1}{2}\fay a_y^2 + \frac{1}{2}\fjy j_y^2 + \frac{1}{2}\fdr d_r^2\dtint{t}\,. \label{eq:J_eingesetzt}
\end{equation}
Die Faktoren \fax,\,\fjx,\,\fay,\,\fjy\,und\fdr\,stehen für Gewichte, mit denen der Einfluss der jeweiligen Größe auf das Gütefunktional variiert werden kann. In dieser Formulierung, das heißt, wenn der Ruck in Längs- wie auch in Querrichtung berücksichtigt werden soll, müssen die Eingangsgrößen des Systems zum Erreichen der entsprechenden Systemordnung zu $\ve{u} = [j_x,\,\dot{\kappa}]^T$ gewählt werden und mit $\kappa$ muss die Krümmung als zusätzlicher Zustand mit der trivialen Bewegungsgleichung $\dot{\kappa} = \dot{\kappa}$ eingeführt werden. Der Zustandsvektor lautet dann $\ve{x} = [s_r,\,v,\,a_x,\,d_r,\,\psi_r,\,\kappa]^T$. Die Querbeschleunigung des Fahrzeugs lässt sich aus der Geschwindigkeit und der Kurvenkrümmung zu 
\begin{equation}
	a_y = \kappa v^2 \label{eq:ay}
\end{equation}
bestimmen \cite{Schramm.2013}. Der Querruck lässt sich durch zeitliches Ableiten von $a_y$ berechnen und lautet
\begin{equation}
j_y = \frac{\textrm{d} a_y}{\textrm{d} t} = \frac{\textrm{d} \kappa v^2}{\textrm{d} t} = \dot{\kappa}v^2 + 2\dot{v}v\kappa  = \dot{\kappa}v^2 + 2a_xv\kappa\,. \label{eq:jy}
\end{equation}
Zur Lösung des Optimierungsproblems mithilfe der indirekten Lösungsmethode kann anschließend die Hamilton-Funktion nach Gleichung \eqref{eq:Hamilton} aufgestellt und partiell nach den einzelnen Systemzuständen zur Herleitung der adjungierten \gls{DGL} abgeleitet werden. Die Hamilton-Funktion lautet
\begin{equation}
\begin{split}
H(\ve{x},\ve{u},\ve{\lambda},t) = \frac{1}{2}\fax a_x^2 + \frac{1}{2}\fjx j_x^2 + \frac{1}{2}\fay (\kappa v^2)^2 + \frac{1}{2}\fjy (\dot{\kappa}v^2 + 2a_xv\kappa)^2 + \frac{1}{2}\fdr d_r^2 + ... \\
... + \frac{\lambda_1v\cos{\psi_r}}{1-d_r\kapparefofs} + \lambda_2a_x + \lambda_3j_x + \lambda_4v\sin{\psi_r} + \lambda_5(\kappa v - \frac{\kapparefofs v\cos{\psi_r}}{1-d_r\kapparefofs}) + \lambda_6\dot{\kappa} \label{eq:Hamilton_eingesetzt}
\end{split}
\end{equation}
und aus Gleichung \eqref{eq:Adjdgl} wiederum folgen die \gls{DGL} für die adjungierten Zustände
\begin{align}
	\begin{split}
		\dot{\lambda_1} &= -\Big(\frac{d_r\dkapparef\lambda_1v\cos{\psi_r}}{(1-d_r\kapparefofs)^2} - \frac{\dkapparef\lambda_5v\cos{\psi_r}}{(1-d_r\kapparefofs)^2}\Big)
	\end{split} \label{eq:dlambda1}
	\\[2ex]
	\begin{split}
		\dot{\lambda_2} &= -\Big(2\fay \kappa^2v^3 + 2\fjy\dot{\kappa}^2v^3 + 4\fjy\kappa^2a_x^2v + 6\fjy\dot{\kappa}\kappa a_xv^2 + ...\\
		&\qquad ... + \frac{\lambda_1\cos{\psi_r}}{1-d_r\kapparefofs} + \lambda_4\sin{\psi_r} + \lambda_5(\kappa - \frac{\kapparefofs \cos{\psi_r}}{1-d_r\kapparefofs})\Big)
	\end{split} \label{eq:dlambda2}
	\\[2ex]
	\begin{split}
		\dot{\lambda_3} &= -\Big(\fax a_x + 4\fjy\kappa^2a_xv^2 + 2\fjy\dot{\kappa}\kappa v^3 + \lambda_2 \Big)
	\end{split} \label{eq:dlambda3}
	\\[2ex]
	\begin{split}
		\dot{\lambda_4} &= -\Big(\fdr d_r + \frac{\kapparefofs\lambda_1v\cos{\psi_r}}{(1-d_r\kapparefofs)^2} -\frac{\kapparefofs^2\lambda_5v\cos{\psi_r}}{(1-d_r\kapparefofs)^2} \Big)
	\end{split} \label{eq:dlambda4}
	\\[2ex]
	\begin{split}
		\dot{\lambda_5} &= -\Big(\lambda_4v\cos{\psi_r} - \frac{\lambda_1v\sin{\psi_r}}{1-d_r\kapparefofs} + \frac{\kapparefofs\lambda_5v\sin{\psi_r}}{1-d_r\kapparefofs}\Big)
	\end{split} \label{eq:dlambda5}
	\\[2ex]
	\begin{split}
		\dot{\lambda_6} &= -\Big(\fay\kappa v^4 + 4\fjy\kappa ax^2v^2 + 2\fjy \dot{\kappa}a_xv^3 + \lambda_5v \Big)\,.
	\end{split} \label{eq:dlambda6}
\end{align}
Mit den \gls{DGL} der adjungierten Zuständen lautet das dynamische Gesamtsystem
\begin{equation}
	\dz = \begin{bmatrix}
	\dx \\
	\dlambda
	\end{bmatrix} =
	\begin{bmatrix}
		\begin{bmatrix}
		\dot{s_r} & \dot{v} & \dot{a_x} & \dot{d_r} & \dot{\psi_r} & \dot{\kappa}
		\end{bmatrix}^T\\
		\begin{bmatrix}
		\dot{\lambda_1} & \dot{\lambda_2} & \dot{\lambda_3} & \dot{\lambda_4} & \dot{\lambda_5} & \dot{\lambda_6}
		\end{bmatrix}^T
	\end{bmatrix}\,.
\end{equation}
Aus der Steuerungsgleichung \eqref{eq:Steuerungsgleichung} folgt 
\begin{align}
j_x &= -\frac{\lambda_3}{\fjx}  \label{eq:jx} \\
\dot{\kappa} &= \frac{-\lambda_6-2\fjy \kappa a_xv^3}{\fjy v^4} \label{eq:dkappa}\,,
\end{align}
Die Randbedingungen \eqref{eq:Anfangswerte}-\eqref{eq:Lambdaend} hängen von der Vorgabe der Anfangs- und Endwerte des jeweiligen Szenarios ab. Für die Transversalitätsbedingung \eqref{eq:Transversalitaet} lässt sich 
\begin{equation}
	H(t_f) = -1
\end{equation}
schreiben, sofern der Endzeitpunkt frei ist und in den Endkosten \Vofxoftf\,ausschließlich $t_f$ bestraft wird.

Die adjungierten \gls{DGL} \eqref{eq:dlambda1}-\eqref{eq:dlambda6} können unter bestimmten Voraussetzungen noch vereinfacht werden. Sollen der Längs- und/oder Querruck nicht berücksichtigt werden, so können die Stellgrößen $\ve{u} = [a_x,\,\kappa]^T$ gewählt werden und das System wird entsprechend um die nicht benötigten adjungierten und Systemzustände reduziert. Ähnliches gilt auch, wenn die Querdynamik vernachlässigt werden soll. Dieser Fall tritt ein, wenn eine reine Geradeausfahrt untersucht wird. Dann kann die Bestrafung der querdynamischen Größen vernachlässigt und das System entsprechend reduziert werden. Die in \ref{subsubsec:Vereinfachungen} angesprochene Annahme konstanter Referenzkrümmung führt dazu, dass die Hamilton-Funktion nicht mehr von $s_r$ abhängt, wodurch die partielle Ableitung verschwindet und damit $\dot{\lambda_1} = 0$ gilt. Es wird deutlich, dass je nach Fahrszenario und Anwendung der Komfortkriterien verschiedene Formulierungen ausgehend von dieser allgemeinen Beschreibung abgeleitet werden können.
