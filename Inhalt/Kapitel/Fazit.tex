\chapter{Fazit und Ausblick}\label{cha:Fazit}
Als Abschluss dieser Arbeit sollen die Ergebnisse und Erkenntnisse nochmals zusammengefasst werden. Anhand dessen soll ein Fazit zu den verwendeten Methoden und Modellbeschreibungen sowie den Komfortkriterien gezogen werden. Zudem sollen weitere relevante Aspekte, die nicht Gegenstand dieser Arbeit waren, als Ausblick für mögliche zukünftige Forschung aufgezeigt werden. 

\section{Zusammenfassung der Ergebnisse}
Das Ziel dieser Arbeit bestand darin, das \gls{AD}-Planungsproblem mithilfe der Variationsrechnung und indirekter Lösungsverfahren für einige selbst gewählte Fahrszenarien zu lösen. Der Lösungsraum sollte dabei analysiert und hinsichtlich des Fahrkomforts interpretiert werden. Zunächst wurde dazu in Kapitel \ref{cha:Komfort} der Begriff Fahrkomfort näher betrachtet und in den Kontext des \gls{AD} eingeordnet. Dabei konnte die besondere Relevanz des Fahrkomforts für die zunehmende Automatisierung von Fahrzeugen und \gls{FAS} hervorgehoben werden. Im Zuge dieser Einordnung wurde außerdem das Phänomen der Kinetose erläutert, wobei festgestellt werden konnte, dass die Vermeidung dieser bei der Weiterentwicklung automatisierter Fahrzeuge im Besonderen beachtet werden muss, weil das Auftreten von Kinetose die \GenderPl{Nutzer}akzeptanz beeinträchtigt. Mithilfe einer umfangreichen Literaturrecherche konnten die Größen Längs- und Querbeschleunigung sowie Längs- und Querruck als maßgebliche und für den Fahrkomfort repräsentative Komfortkriterien herausgearbeitet werden. Gleichzeitig wurde aber auch die Schwierigkeit dargelegt, objektive und allgemeingültige Grenzwerte für die Komfortkriterien anzugeben, da der erlebte Fahrkomfort zum einen situationsabhängig ist und zum anderen stark dem individuellen und subjektiven Empfinden jeder Person unterliegt. Aus diesem Grund sind die Werte, die im Anhang \ref{app:Tabelle} in den Tabellen \ref{tab:Komfortwerte_längs} und \ref{tab:Komfortwerte_quer} zusammengetragen wurden, auch eher als Richtwerte denn als harte Komfortgrenzen zu verstehen. Als weiteres Komfortkriterium wurde die Reisezeit ermittelt, da eine geringe Reisezeit oftmals höhere Beschleunigungs- und Ruckwerte rechtfertigt. In Kapitel \ref{cha:Optimierung} wurden mithilfe der Variationsrechnung notwendige Optimalitätsbedingungen zur Lösung eines dynamischen \gls{OP} hergeleitet. Die Diskussion verschiedener indirekter Lösungsverfahren ergab, dass die indirekten Kollokationsverfahren für die Lösung des \gls{AD}-Planungsproblems besonders geeignet sind, weil diese robuste Lösungen hoher Genauigkeit liefern und für eine große Klasse von Randwertproblemen einsetzbar sind. In Kapitel \ref{cha:Modellbildung} wurde ein kinematisches Einspurmodell zur Formulierung der Fahrzeugbewegung in Frenet-Koordinaten entlang einer gegebenen Referenzkurve hergeleitet, da diese Herangehensweise zur Fahrzeugmodellierung einen bewährten Ansatz in der Trajektorienplanung für automatisierte Fahrzeuge darstellt. Dieses Fahrzeugmodell wurde in Kapitel \ref{cha:Ergebnisse} zur Lösung dynamischer \gls{OP} für verschiedene Fahrszenarien unter Berücksichtigung der in Kapitel \ref{cha:Komfort} erarbeiteten Komfortkriterien verwendet. Unter Vernachlässigung der Querdynamik konnte so der Lösungsraum einer reinen Geradeausfahrt für zwei unterschiedliche Gütefunktionale bestimmt werden. Bei der reinen Bestrafung der Stellgröße besteht der Lösungsraum der kinematischen Größen aus Polynomen. Im Fall, dass zusätzlich zum Längsruck auch die -beschleunigung bestraft wird, besteht die Lösung neben einem polynomialen Anteil aus einem stabilen und einem instabilen Exponentialanteil. Mithilfe dieser Lösung konnte nachgewiesen werden, dass die Längsbeschleunigung bei der Geradeausfahrt durch den linearen Anteil der Lösung beschränkt ist und so in Abhängigkeit von den Randwerten und Gewichtungsfaktoren eine obere und untere Grenze der zu erwartenden Beschleunigung angegeben werden kann. Des Weiteren wurde die Geradeausfahrt auf das Szenario Heranfahren an eine Ampel bei bekannter Rotphase erweitert. Dabei wurden zwei Planungsstrategien, die einmal der menschlichen, nicht vorausschauenden Planung und einmal der maschinellen, vorausschauenden Planung entsprechen, miteinander verglichen. Es wurde festgestellt, dass die vorausschauende Planung zwar sowohl komfortablere als auch schnellere Trajektorien liefert, jedoch nicht in allen Verkehrssituationen praktikabel ist, da ein frühzeitiges Abbremsen den nachfolgenden Verkehr ausbremsen würde und das Fahrzeug bei Stau vor der Ampel keine freie Fahrt zum uneingeschränkten Beschleunigen hat. Anschließend wurden unter Hinzunahme der Querdynamik Spurwechsel untersucht. Dabei konnte festgestellt werden, dass sich der Fahrkomfort über die Gewichtung der Querabweichung zur Referenzkurve beeinflussen lässt. Eine stärkere Bestrafung der seitlichen Abweichung führt zu einem \glqq strengeren\grqq~Folgen der Referenzkurve, woraus unkomfortablere Trajektorien für die Querbeschleunigung und den Querruck resultieren. Eine geringere Gewichtung hingegen führt zu weicheren und damit komfortableren Trajektorien. Außerdem konnte gezeigt werden, dass sich die Quer- und Längsdynamik bei Trajektorien geringer Querdynamik getrennt voneinander betrachten lassen. Das Szenario Kurvenfahrt wurde zunächst durch eine reine Kreisfahrt mit konstanter Referenzkrümmung approximiert. Die Untersuchung des um $s_r$ reduzierten Fahrzeugsystems ergab, dass bei der Kreisfahrt eine Ruhelage existiert, die eine Lösung des dynamischen \gls{OP} darstellt. Außerdem konnte eine gewöhnliche, nichtlineare \gls{DGL} zweiter (bzw. vierter) Ordnung zur Bestimmung der optimalen Geschwindigkeitstrajektorie hergeleitet werden. Anschließend wurde die Formulierung der Kurvenfahrt auf eine klothoidenförmige Kurve mit veränderlicher Krümmung erweitert. Obwohl bei dieser Formulierung der Referenzkurve keine echte Ruhelage existiert, konnte trotzdem gezeigt werden, dass die optimalen Trajektorien punktweise gegen die krümmungsabhängige Ruhelage der Kreisfahrt konvergieren, sodass die Ruhelage der Kreisfahrt als Näherungslösung für lange Klothoiden verwendet werden kann. Außerdem wurde festgestellt, dass die Anwendung des Maximumprinzips auf die Stellgröße der Querrichtung $\dot{\kappa}$ zwar keine echte Beschränkung für den Querruck im Sinne harter Grenzwerte darstellt, jedoch durchaus eine Reduktion von $j_y$ durch die Beschränkung von $\dot{\kappa}$ möglich ist. Schließlich wurden die Szenarien der Geradeaus- und Kurvenfahrt zu einer Gerade-Kurve-Gerade-Kombination verknüpft und es konnte gezeigt werden, dass sich die Ergebnisse der einzelnen Teilszenarien auch in diesem Gesamtszenario wiederfinden. Während die Lösungen im mittleren Kurvenabschnitt gegen die (punktweise) Ruhelage der Kurve konvergieren, entsprechen sie in den Geradenabschnitten den zuvor ermittelten Lösungen der Geradeausfahrt. Außerdem wurden die Vorteile der Beschreibung einer Kurve als Klothoide speziell an den Übergangspunkten zwischen Geraden- und Kurvenstücken gegenüber der Beschreibung als Kreis hinsichtlich des Fahrkomforts dargelegt. Die stetige Krümmungsänderung implizitert bereits ruckoptimale Trajektorien, sodass der Fahrkomfort speziell an den Übergängen zwischen Geraden- und Kurvenstücken durch die Anwendung klothoidenförmiger Referenzkurven deutlich gesteigert werden kann. 

Insgesamt lässt sich festhalten, dass die Variationsrechnung eine geeignete Herangehensweise zur Lösung dynamischer \gls{OP} dargestellt und auch komplexe Fahrzenarien, die sich aus mehreren Teilszenarien zusammensetzen, mithilfe der verwendeten Methoden gelöst werden können. Dabei kann der große Vorteil der Variationsrechnung, tiefe Einblicke in die Struktur der optimalen Trajektorien zu erhalten, in hohem Maße ausgenutzt werden. 
%In diesem Kapitel wurden außerdem verschiedene indirekte Lösungsverfahren vorgestellt und hinsichtlich ihrer Eignung zur Lösung des \gls{AD}-Planungsproblems diskutiert und in den Kontext \gls{AD} eingeordnet. In Kapitel \ref{cha:Modellbildung} wurde ein kinematisches Einspurmodell zur Formulierung der Fahrzeugbewegung in Frenet-Koordinaten entlang einer gegebenen Referenzkurve hergeleitet. Diese Herangehensweise zur Fahrzeugmodellierung stellt einen bewährten Ansatz in der Trajektorienplanung für automatisierte Fahrzeuge dar. Im Zuge dessen wurde die Parametrierung zur Beschreibung der Referenzkurve sowie einige relevante Vereinfachungen und Annahmen erläutert. Anschließend wurden die in Kapitel \ref{cha:Optimierung} hergeleiteten Optimalitätsbedingungen auf das Fahrzeugmodell unter Berücksichtigung aller Größen angewendet und das daraus resultierende kanonische \gls{DGL}-System aufgestellt. In Kapitel \ref{cha:Ergebnisse} wurden dynamische \gls{OP} für verschiedene Fahrszenarien unter Berücksichtigung der in Kapitel \ref{cha:Komfort} erarbeiteten Komfortkriterien formuliert und mithilfe der Methoden aus Kapitel \ref{cha:Optimierung} gelöst. Je nach Szenario und Modellkomplexität konnten einige \gls{OP} analytisch gelöst und die Lösung vollständig berechnet werden. Unter Vernachlässigung der Querdynamik konnte so beispielsweise der Lösungsraum einer reinen Geradeausfahrt bestimmt werden. Diese Erkenntnis wurde verwendet, um einfache Abschätzungen für die komfortrelevante Längsbeschleunigung in Abhängigkeit der Randwerte und Gewichtungsfaktoren anzugeben. Beschränkungen der Stellgröße, die dem Längsruck des Fahrzeugs entspricht, ließen sich mithilfe des Maximumprinzips berücksichtigen. Des Weiteren wurde das Szenario Heranfahren an eine Ampel bei bekannter Rotphase untersucht. Dabei wurden zwei Planungsstrategien, die einmal der menschlichen, nicht vorausschauenden Planung und einmal der maschinellen, vorausschauenden Planung entsprechen, miteinander verglichen. Es wurde festgestellt, dass die vorausschauende Planung zwar sowohl komfortablere als auch schnellere Trajektorien liefert, jedoch nicht in allen Verkehrsituationen praktikabel ist. Anschließend wurden Spurwechsel untersucht und es konnte gezeigt werden, dass sich der Fahrkomfort über die Gewichtung der Querabweichung zur Referenzkurve beeinflussen lässt. Das Szenario Kurvenfahrt wurde zunächst durch eine reine Kreisfahrt mit konstanter Referenzkrümmung approximiert und es konnte gezeigt werden, dass eine Ruhelage des um $s_r$ reduzierten Fahrzeugsystems existiert, die eine Lösung des dynamischen \gls{OP} darstellt. Diese Ruhelage findet sich bei der Formulierung der Kurvenfahrt mithilfe einer Klothoide wieder, deren Referenzkrümmung streckenabhängig ist. Auch wenn bei der Betrachtung der Kurve in klothoidenform streng genommen keine Ruhelage existiert, so wurde doch festgestellt, dass die optimalen Trajektorien punktweise gegen die krümmungsabhängige Ruhelage der Kreisfahrt konvergieren. Schließlich wurden die Szenarien der Geradeaus- und Kurvenfahrt zu einer Gerade-Kurve-Gerade-Kombination verknüpft und es konnte gezeigt werden, dass sich die Ergebnisse der einzelnen Teilszenarien auch in diesem Gesamtszenario wiederfinden. Außerdem wurden die Vorteile der Beschreibung einer Kurve als Klothoide speziell an den Übergangspunkten zwischen Geraden- und Kurvenstücken gegenüber der Beschreibung als Kreis hinsichtlich des Fahrkomforts dargelegt.
\section{Ausblick auf zukünftige Forschungsarbeit}
Die zukünftige Forschungsarbeit zu diesem Thema lässt sich unterteilen in weitere Forschung auf der Softwareebene, so wie sie in dieser Arbeit durchgeführt wurde, und die praktische Anwendbarkeit der Erkenntnisse, die aus dieser Arbeit abgeleitet werden können. Die Ergebnisse dieser Arbeit resultieren aus speziellen, selbst gewählten Fahrszenarien. Die Anwendung der Optimalitätsbedingungen auf das jeweilige Szenario und die Herleitung des entsprechenden Randwertproblems wurden dabei für jedes Szenario händisch durchgeführt. Daher wäre eine Automatisierung der Lösung für generische Fahrszenarien und Einbettung in einen (echtzeitfähigen) Algorithmus ein denkbarer Ansatz für die softwareseitige Weiterentwicklung. In dieser Arbeit wurde zum Teil der Einfluss unterschiedlicher Gewichtungsfaktoren auf den Fahrkomfort untersucht. Allerdings wurden die Gewichtungsfaktoren für jede Lösung vor Beginn einer Optimierung festgelegt und im weiteren Verlauf nicht mehr verändert. Analog zur Berücksichtigung unterschiedlicher Referenzkrümmungen mithilfe von diskontinuierlichen Systemdynamiken wäre eine Variation der Gewichtungsfaktoren denkbar. Dadurch könnte eine veränderliche Gewichtung einzelner Komfortkriterien untersucht werden. Ein ähnlicher Ansatz wurde bereits in Abschnitt \ref{sec:Spurwechsel} zur Einbettung einer Sicherheitsbewertung bei Spurwechseln vorgeschlagen. In Kapitel \ref{cha:Komfort} wurde unter anderem die Bedeutung von Kinetose hervorgehoben. Die Hinzunahme eines mathematischen Modells zur Modellierung von Kinetose, wie es in \cite{Kamiji.17.09.200720.09.2007} vorgestellt wird, bietet die Möglichkeit zur zusätzlichen Berücksichtigung der \gls{MSI} (vgl. \cite{OHanlon.1974}).

Die Bewertung des Fahrkomforts basiert in dieser Arbeit ausschließlich auf den in Kapitel \ref{cha:Komfort} mithilfe einschlägiger Literatur erarbeiteten Grenz- und Richtwerten. Eine Bewertung der Eignung der Komfortkriterien und der verwendeten Gütefunktionale ist daher nur eingeschränkt möglich. Aus diesem Grund stellen die Anwendung der in dieser Arbeit (oder zumindest nach der selben Vorgehensweise) geplanten Trajektorien in realen Fahrszenarien und eine anschließende Bewertung des erlebten Fahrkomforts durch \GenderPl{Proband} eine interessante Möglichkeit zur Bewertung des tatsächlich erlebten Fahrkomforts dar. Ebenso ist ein Vergleich der in dieser Arbeit geplanten Trajektorien mit menschlichen Fahrtrajektorien denkbar. Dabei könnte untersucht werden, ob die Verläufe insbesondere qualitativ deckungsgleich sind. Beispielsweise könnte überprüft werden, ob auch ein menschlicher Fahrer oder eine menschliche Fahrerin bei einer Kreisfahrt eine Trajektorie wählen würde, die gegen die Ruhelage konvergiert. In dieser Arbeit konnten mithilfe der Variationsrechnung und Lösung dynamischer \gls{OP} mittels indirekter Verfahren tiefgehende Erkenntnisse und ein analytisches Verständnis für die Struktur komfortoptimaler Trajektorien auch für komplexe Fahrszenarien gewonnen werden. Aufgrund der gesellschaftlichen Relevanz, die das \gls{AD} besitzt, sollten die Ergebnisse dieser Arbeit in weiterer Forschung vertieft werden. 