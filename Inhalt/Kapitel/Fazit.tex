\chapter{Fazit und Ausblick}\label{cha:Fazit}
Als Abschluss dieser Arbeit lässt sich festhalten, dass mit den Szenarien Geradeausfahrt, Spurwechsel, kreisförmige und klothoidenförmige Kurvenfahrt sowie der Gerade-Kurve-Gerade-Kombination verschiedene Fahrszenarien analysiert und hinsichtlich des Fahrkomforts interpretiert werden konnten. Die verwendeten Komfortkriterien Zeit, Längs- und Querbeschleunigung sowie Längs- und Querruck stellen dabei in der Forschung häufig eingesetzte Gütekriterien zur Bewertung des Fahrkomforts dar. Es wurden mehrere Systembeschreibungen erklärt, die sich in der Modellierung der Querdynamik sowie den verwendeten Stellgrößen unterscheiden und verschiedene Möglichkeiten zur Beschreibung der Referenzkurve dargelegt. Die System- und Referenzbeschreibungen wurden bezüglich ihrer Eignung für die einzelnen Fahrszenarien für die komfortbasierte Trajektorienplanung untersucht und es wurden jeweils Vor- und Nachteile hinsichtlich des Fahrkomforts erarbeitet. 