\newcommand{\eqp}{\ensuremath{\, \, \, .}}
\newcommand{\M}[1]{\textbf{#1}} %Matrix  M
\newcommand{\Mr}[1]{\textbf{#1}_\rho} %Matrix mit tiefgestelltem rho, M_rho
\newcommand{\Mtil}[1]{\tilde{\textbf{#1}}} %Matrix mit Tilde
\newcommand{\Mtilt}[2]{\tilde{\textbf{#1}}_{#2}} %Matrix mit Tilde mit tiefgestelltem arg2
\DeclareRobustCommand{\Mt}[2]{\textbf{#1}_{#2}} %Matrix M mit tiefgestelltem arg2


\DeclareRobustCommand{\w}[1]{\underline{#1}} %Vektor arg1 unterstrichen 
\DeclareRobustCommand{\wt}[2]{\underline{#1}_{#2}} %Vektor unterstrichen w mit tiefgestelltem arg2
\DeclareRobustCommand{\wr}[1]{\underline{#1}_{\rho}} %Vektor unterstrichen mit rho, w_rho

% Textbausteine
% =============
	% Produktnamen
	\newcommand*{\Matlab}{\textsc{Matlab}}
	\newcommand*{\Matlabreg}{\textsc{Matlab}\textsuperscript{\tiny \textregistered}}
	\newcommand*{\MatSim}{\textsc{Matlab/Simulink}}
	\newcommand*{\Simulink}{\textsc{Simulink}}
	\newcommand*{\Simulinkreg}{\textsc{Simulink}\textsuperscript{\tiny \textregistered}}

	% Das Makro |\name|\marg{person} formatiert einen Personennamen bspw. eines Erfinders oder Entdeckers gemäß |\name{Euler}| \arrow\ \name{Euler}.
	\newcommand*{\name}[1]{\textsc{#1}}
	\newcommand*{\GenderPl}[1]{{#1}\_innen}
	\newcommand*{\GendeSin}[1]{{#1}\_in}
% Makros für Einheiten, Exponenten
% ================================

	\newcommand*{\unit}[1]{\ensuremath{\mathrm{#1}}}
	
	% Wert mit Einheit (mit kleinem Leerzeichen dazwischen), aus Text- UND Math-Modus
	\newcommand*{\valunit}[2]{\ensuremath{#1\,\mrm{#2}}}
	
	
	% "°C", im Text- oder Mathe-Modus
	\newcommand*{\degC}{
		\ifmmode
		^\circ \mrm{C}%
		\else
		\textdegree C%
		\fi}
	
	\newcommand*{\degree}{
		\ifmmode
		^\circ%
		\else
		\textdegree%
		\fi}
	
	% Für Exponentenschreibweise ( Anwendung: 123\E{3} )
	\newcommand*{\E}[1]{\ensuremath{\cdot 10^{#1}}}
	
	\newcommand*{\eexp}[1]{\ensuremath{\mathrm{e}^{#1}}}
	\newcommand*{\iu}{\ensuremath{\mathrm{j}}}
	
	\newcommand*{\todots}{\ensuremath{,\,\hdots,\,}}

% Makros für Formeln
% ==================

	% Definition für Vektor und Matizen
	\newcommand*{\mat}[1]{{\ensuremath{\boldsymbol{\mathrm{#1}}}}}
	\newcommand*{\ma}[1]{{\ensuremath{\boldsymbol{\mathrm{#1}}}}}
	\newcommand*{\mas}[1]{\ensuremath{\boldsymbol{#1}}}
	\newcommand*{\ve}[1]{\ensuremath{\boldsymbol{#1}}}
	\newcommand*{\ves}[1]{\ensuremath{\boldsymbol{\mathrm{#1}}}}
	
	\newcommand*{\AP}{\ensuremath{\mathrm{AP}}}
	\newcommand*{\doti}{\ensuremath{(i)^\cdot}}
	
	\newcommand*{\inprod}[2]{\ensuremath{\langle #1,\,#2 \rangle}}
	
	\newcommand*{\ul}[1]{\underline{#1}}
	
	% gerades "d" (z.B. für Integral)
	\newcommand*{\ud}{\ensuremath{\mathrm{d}}}
	
	% normaler Text in Formeln
	\newcommand*{\tn}[1]{\textnormal{#1}}
	
	% nicht-kursive Schrift in Formeln
	\newcommand*{\mrm}[1]{\ensuremath{\mathrm{#1}}}
	
	% gerades "T" für Transponiert
	\newcommand*{\transp}{\ensuremath{\mathrm{T}}}
	
	% gerades "rg"
	\newcommand*{\rang}{\ensuremath{\operatorname{rg}}}
	
	% Für geklammerte Ausdrücke mit Index (Subscript)
	% (einmal mit kursiven Index, einmal mit geradem Index)
	\newcommand*{\grpsb}[2]{\ensuremath{\left(#1\right)_{#2}}}
	\newcommand*{\grprsb}[2]{\ensuremath{\left(#1\right)_{\mathrm{#2}}}}
	
	% Ableitungen und Integrale
		% "normale" Ableitung (mit geraden "d"s)
		\newcommand*{\normd}[2]{\ensuremath{\frac{\mathrm{d}#1}{\mathrm{d}#2}}}
		\newcommand*{\normdat}[3]{\ensuremath{\left.\frac{\mathrm{d} #1}{\mathrm{d} #2}\right|_{#3}}}
		
		% Materielle Ableitung
		\newcommand*{\matd}[2]{\ensuremath{\frac{\mathrm{D} #1}{\mathrm{D} #2}}}
		\newcommand*{\matdat}[3]{\ensuremath{\left.\frac{\mathrm{D} #1}{\mathrm{D} #2}\right|_{#3}}}
		
		% Partielle Ableitung
		\newcommand*{\partiald}[2]{\ensuremath{\frac{\partial #1}{\partial #2}}}
		\newcommand*{\partialdat}[3]{\ensuremath{\left.\frac{\partial #1}{\partial #2}\right|_{#3}}}
	
	
	% Transformationen
	\newcommand*{\FT}[1]{\ensuremath{\mathfrak{F}\left\{#1\right\}}}
	\newcommand*{\FTabs}[1]{\ensuremath{\left|\mathfrak{F}\left\{#1\right\}\right|}}
	\newcommand*{\IFT}[1]{\ensuremath{\mathfrak{F}^{-1}\left\{#1\right\}}}
	\newcommand*{\DFT}[1]{\ensuremath{\mathrm{DFT}\left\{#1\right\}}}
	\newcommand*{\DFTabs}[1]{\ensuremath{\left|\mathrm{DFT}\left\{#1\right\}\right|}}
	\newcommand*{\Laplace}[1]{\ensuremath{\mathfrak{L}\left(#1\right)}}
	\newcommand*{\InvLaplace}[1]{\ensuremath{\mathfrak{L^{-1}}\left(#1\right)}}
	\newcommand*{\invtrans}{\ensuremath{\quad\bullet\!\!-\!\!\!-\!\!\circ\quad}}
	\newcommand*{\trans}{\ensuremath{\quad\circ\!\!-\!\!\!-\!\!\bullet\quad}}
	
	
	\newcommand*{\mlfct}[1]{\texttt{#1}}
	\newcommand*{\mlvar}[1]{\texttt{#1}}
	
	
	% Manche textcomp-Zeichen funktionieren mit dem TU-Design nicht, diese können dann mit diesem
	% Befehl gesetzt werden.
	\newcommand*{\textcompstdfont}[1]{{\fontfamily{cmr} \fontseries{m} \fontshape{n} \selectfont #1}}

% Makros für Variablen
% ================================
\newcommand*{\amax}{\ensuremath{a_{\textrm{max}}}}
\newcommand*{\amin}{\ensuremath{a_{\textrm{min}}}}
\newcommand*{\jmax}{\ensuremath{j_{\textrm{max}}}}
\newcommand*{\jmin}{\ensuremath{j_{\textrm{min}}}}
\newcommand*{\opt}[1]{\ensuremath{{#1}^*}}
\newcommand*{\fofx}{\ensuremath{f(\ve{x})}}
\newcommand*{\J}{\ensuremath{J(\ve{x}(t),\ve{u}(t),t)}}
\newcommand*{\Vofxoftf}{\ensuremath{V(\ve{x}(t_f),t_f)}}
\newcommand*{\Vofxoftone}{\ensuremath{\tilde{V}(\ve{x}(t_1),t_1)}}
\newcommand*{\Vofxoftfallgemein}{\ensuremath{\bar{V}(\ve{x}(t_f),t_f)}}
\newcommand*{\Vofxoftoftonefallgemein}{\ensuremath{\hat{V}(\ve{x}(t_f),\ve{x}(t_1),t_f,t_1)}}
\newcommand*{\variation}[1]{\ensuremath{\ve{\delta}_{#1}}}
\newcommand*{\dtint}[1]{\ensuremath{\textrm{d}{#1}}}
% Makros für x(t) Variablen
% ================================
\newcommand*{\xzero}{\ensuremath{\ve{x}_0}}
\newcommand*{\xoft}{\ensuremath{\ve{x}(t)}}
\newcommand*{\xoftf}{\ensuremath{\ve{x}(t_f)}}
\newcommand*{\xoftzero}{\ensuremath{\ve{x}(t_0)}}
\newcommand*{\xoftone}{\ensuremath{\ve{x}(t_1)}}
\newcommand*{\xofti}{\ensuremath{\ve{x}(t_i)}}
\newcommand*{\xoftoneminus}{\ensuremath{\ve{x}(t_1^-)}}
\newcommand*{\xoftoneplus}{\ensuremath{\ve{x}(t_1^+)}}
\newcommand*{\xoptoft}{\ensuremath{\opt{\ve{x}}(t)}}
\newcommand*{\xoptoftf}{\ensuremath{\opt{\ve{x}}(t_f)}}
\newcommand*{\xoptoftzero}{\ensuremath{\opt{\ve{x}}(t_0)}}
\newcommand*{\xoptoftfopt}{\ensuremath{\opt{\ve{x}}(\opt{t_f})}}
% Makros für x(tau) Variablen
% ================================
\newcommand*{\xoftau}{\ensuremath{\ve{x}(\tau)}}
\newcommand*{\xoftauf}{\ensuremath{\ve{x}(\tau_f)}}
\newcommand*{\xoftauzero}{\ensuremath{\ve{x}(\tau_0)}}
\newcommand*{\xoptoftau}{\ensuremath{\opt{\ve{x}}(t\tau)}}
\newcommand*{\xoptoftauf}{\ensuremath{\opt{\ve{x}}(\tau_f)}}
\newcommand*{\xoptoftauzero}{\ensuremath{\opt{\ve{x}}(\tau_0)}}
\newcommand*{\xoptoftaufopt}{\ensuremath{\opt{\ve{x}}(\opt{\tau_f})}}
% Makros für dx(t) Variablen
% ================================
\newcommand*{\dx}{\ensuremath{\dot{\ve{x}}}}
\newcommand*{\dxoft}{\ensuremath{\dot{\ve{x}}(t)}}
\newcommand*{\dxoftf}{\ensuremath{\dot{\ve{x}}(t_f)}}
\newcommand*{\dxoftzero}{\ensuremath{\dot{\ve{x}}(t_0)}}
\newcommand*{\dxoptoft}{\ensuremath{\dot{\ve{x}}^*(t)}}
\newcommand*{\dxoptoftf}{\ensuremath{\dot{\ve{x}}^*(t_f)}}
\newcommand*{\dxoptoftfopt}{\ensuremath{\dot{\ve{x}}^*(\opt{t_f})}}
% Makros für dx(tau) Variablen
% ================================
\newcommand*{\dxoftau}{\ensuremath{\ve{x}'(\tau)}}
\newcommand*{\dxoftauf}{\ensuremath{\ve{x}'(\tau_f)}}
\newcommand*{\dxoftauzero}{\ensuremath{\ve{x}'(\tau_0)}}
\newcommand*{\dxoptoftau}{\ensuremath{\ve{x}'^*(\tau)}}
\newcommand*{\dxoptoftauf}{\ensuremath{\ve{x}'^*(\tau_f)}}
\newcommand*{\dxoptoftaufopt}{\ensuremath{\ve{x}'^*(\opt{\tau_f})}}
% Makros für u(t) Variablen
% ================================
\newcommand*{\uoft}{\ensuremath{\ve{u}(t)}}
\newcommand*{\uoftf}{\ensuremath{\ve{u}(t_f)}}
\newcommand*{\uoftzero}{\ensuremath{\ve{u}(t_0)}}
\newcommand*{\uoptoft}{\ensuremath{\opt{\ve{u}}(t)}}
\newcommand*{\uoptoftf}{\ensuremath{\opt{\ve{u}}(t_f)}}
\newcommand*{\uoptoftfopt}{\ensuremath{\opt{\ve{u}}(\opt{t_f})}}
% Makros für u(tau) Variablen
% ================================
\newcommand*{\uoftau}{\ensuremath{\ve{u}(\tau)}}
\newcommand*{\uoftauf}{\ensuremath{\ve{u}(\tau_f)}}
\newcommand*{\uoftauzero}{\ensuremath{\ve{u}(\tau_0)}}
\newcommand*{\uoptoftau}{\ensuremath{\opt{\ve{u}}(\tau)}}
\newcommand*{\uoptoftauf}{\ensuremath{\opt{\ve{u}}(\tau_f)}}
\newcommand*{\uoptoftaufopt}{\ensuremath{\opt{\ve{u}}(\opt{\tau_f})}}
% Makros für lambda(t) Variablen
% ================================
\newcommand*{\lambdaoft}{\ensuremath{\ve{\lambda}(t)}}
\newcommand*{\lambdaoftf}{\ensuremath{\ve{\lambda}(t_f)}}
\newcommand*{\lambdaoftone}{\ensuremath{\ve{\lambda}(t_1)}}
\newcommand*{\lambdaoftzero}{\ensuremath{\ve{\lambda}(t_0)}}
\newcommand*{\lambdaoftoneminus}{\ensuremath{\ve{\lambda}(t_1^-)}}
\newcommand*{\lambdaoftoneplus}{\ensuremath{\ve{\lambda}(t_1^+)}}
% Makros für lambda(tau) Variablen
% ================================
\newcommand*{\lambdaoftau}{\ensuremath{\ve{\lambda}(\tau)}}
\newcommand*{\lambdaoftauf}{\ensuremath{\ve{\lambda}(\tau_f)}}
% Makros für dlambda(t) Variablen
% ================================
\newcommand*{\dlambda}{\ensuremath{\dot{\ve{\lambda}}}}
\newcommand*{\dlambdaoft}{\ensuremath{\dot{\ve{\lambda}}(t)}}
% Makros für dlambda(tau) Variablen
% ================================
\newcommand*{\dlambdaoftau}{\ensuremath{\ve{\lambda}'(\tau)}}
% Makros für z(t) Variablen
% ================================
\newcommand*{\zzero}{\ensuremath{\ve{z}_0}}
\newcommand*{\zoft}{\ensuremath{\ve{z}(t)}}
\newcommand*{\zoftf}{\ensuremath{\ve{z}(t_f)}}
\newcommand*{\zoftzero}{\ensuremath{\ve{z}(t_0)}}
\newcommand*{\zoptoft}{\ensuremath{\opt{\ve{z}}(t)}}
\newcommand*{\zoptoftf}{\ensuremath{\opt{\ve{z}}(t_f)}}
\newcommand*{\zoptoftzero}{\ensuremath{\opt{\ve{z}}(t_0)}}
\newcommand*{\zoptoftfopt}{\ensuremath{\opt{\ve{z}}(\opt{t_f})}}
% Makros für z(tau) Variablen
% ================================
\newcommand*{\zoftau}{\ensuremath{\ve{z}(\tau)}}
\newcommand*{\zoftauf}{\ensuremath{\ve{z}(\tau_f)}}
\newcommand*{\zoftauzero}{\ensuremath{\ve{z}(\tau_0)}}
\newcommand*{\zoptoftau}{\ensuremath{\opt{\ve{z}}(\tau)}}
\newcommand*{\zoptoftauf}{\ensuremath{\opt{\ve{z}}(\tau_f)}}
\newcommand*{\zoptoftauzero}{\ensuremath{\opt{\ve{z}}(\tau_0)}}
\newcommand*{\zoptoftaufopt}{\ensuremath{\opt{\ve{z}}(\opt{\tau_f})}}
% Makros für dz(t) Variablen
% ================================
\newcommand*{\dz}{\ensuremath{\dot{\ve{z}}}}
\newcommand*{\dzoft}{\ensuremath{\dot{\ve{z}}(t)}}
\newcommand*{\dzoftf}{\ensuremath{\dot{\ve{z}}(t_f)}}
\newcommand*{\dzoftzero}{\ensuremath{\dot{\ve{z}}(t_0)}}
\newcommand*{\dzoptoft}{\ensuremath{\dot{\ve{z}}^*(t)}}
\newcommand*{\dzoptoftf}{\ensuremath{\dot{\ve{z}}^*(t_f)}}
\newcommand*{\dzoptoftfopt}{\ensuremath{\dot{\ve{z}}^*(\opt{t_f})}}
% Makros für dz(tau) Variablen
% ================================
\newcommand*{\dzoftau}{\ensuremath{\ve{z}'(\tau)}}
\newcommand*{\dzoftauf}{\ensuremath{\ve{z}'(\tau_f)}}
\newcommand*{\dzoftauzero}{\ensuremath{\ve{z}'(\tau_0)}}
\newcommand*{\dzoptoftau}{\ensuremath{\ve{z}'^*(\tau)}}
\newcommand*{\dzoptoftauf}{\ensuremath{\ve{z}'^*(\tau_f)}}
\newcommand*{\dzoptoftaufopt}{\ensuremath{\ve{z}'^*(\opt{\tau_f})}}
% Makros für Ttransformation Variablen
% ================================
\newcommand*{\Ttransoftau}{\ensuremath{\mathcal{T}(\tau)}}
\newcommand*{\Ttransoftauzero}{\ensuremath{\mathcal{T}(\tau_0)}}
\newcommand*{\Ttransoftauf}{\ensuremath{\mathcal{T}(\tau_f)}}
% Makros für dTtransformation Variablen
% ================================
\newcommand*{\dTtransoftau}{\ensuremath{\mathcal{T}'(\tau)}}
\newcommand*{\dTtransoftauzero}{\ensuremath{\mathcal{T}'(\tau_0)}}
\newcommand*{\dTtransoftauf}{\ensuremath{\mathcal{T}'(\tau_f)}}

% Makros für Frenet Variablen
% ================================
\newcommand*{\kapparef}{\ensuremath{\kappa_{\textrm{ref}}}}
\newcommand*{\kapparefzero}{\ensuremath{\kappa_{\textrm{ref},0}}}
\newcommand*{\kapparefofs}{\ensuremath{\kappa_{\textrm{ref}}(s_r)}}
\newcommand*{\dkapparef}{\ensuremath{\kappa'_{\textrm{ref}}}}
\newcommand*{\psiref}{\ensuremath{\psi_{\textrm{ref}}}}

% Makros für Gewichte
% ================================
\newcommand*{\fax}{\ensuremath{f_{a_x}}}
\newcommand*{\fay}{\ensuremath{f_{a_y}}}
\newcommand*{\fjx}{\ensuremath{f_{j_x}}}
\newcommand*{\fjy}{\ensuremath{f_{j_y}}}
\newcommand*{\fdr}{\ensuremath{f_{d_r}}}