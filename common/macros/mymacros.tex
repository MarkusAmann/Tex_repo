\newcommand{\eqp}{\ensuremath{\, \, \, .}}
\newcommand{\M}[1]{\textbf{#1}} %Matrix  M
\newcommand{\Mr}[1]{\textbf{#1}_\rho} %Matrix mit tiefgestelltem rho, M_rho
\newcommand{\Mtil}[1]{\tilde{\textbf{#1}}} %Matrix mit Tilde
\newcommand{\Mtilt}[2]{\tilde{\textbf{#1}}_{#2}} %Matrix mit Tilde mit tiefgestelltem arg2
\DeclareRobustCommand{\Mt}[2]{\textbf{#1}_{#2}} %Matrix M mit tiefgestelltem arg2


\DeclareRobustCommand{\w}[1]{\underline{#1}} %Vektor arg1 unterstrichen 
\DeclareRobustCommand{\wt}[2]{\underline{#1}_{#2}} %Vektor unterstrichen w mit tiefgestelltem arg2
\DeclareRobustCommand{\wr}[1]{\underline{#1}_{\rho}} %Vektor unterstrichen mit rho, w_rho

% Textbausteine
% =============
	% Produktnamen
	\newcommand*{\Matlab}{\textsc{Matlab}}
	\newcommand*{\Matlabreg}{\textsc{Matlab}\textsuperscript{\tiny \textregistered}}
	\newcommand*{\MatSim}{\textsc{Matlab/Simulink}}
	\newcommand*{\Simulink}{\textsc{Simulink}}
	\newcommand*{\Simulinkreg}{\textsc{Simulink}\textsuperscript{\tiny \textregistered}}

	% Das Makro |\name|\marg{person} formatiert einen Personennamen bspw. eines Erfinders oder Entdeckers gemäß |\name{Euler}| \arrow\ \name{Euler}.
	\newcommand*{\name}[1]{\textsc{#1}}
	\newcommand*{\GenderPl}[1]{{#1}\_innen}
	\newcommand*{\GendeSin}[1]{{#1}\_in}
% Makros für Einheiten, Exponenten
% ================================

	\newcommand*{\unit}[1]{\ensuremath{\mathrm{#1}}}
	
	% Wert mit Einheit (mit kleinem Leerzeichen dazwischen), aus Text- UND Math-Modus
	\newcommand*{\valunit}[2]{\ensuremath{#1\,\mrm{#2}}}
	
	
	% "°C", im Text- oder Mathe-Modus
	\newcommand*{\degC}{
		\ifmmode
		^\circ \mrm{C}%
		\else
		\textdegree C%
		\fi}
	
	\newcommand*{\degree}{
		\ifmmode
		^\circ%
		\else
		\textdegree%
		\fi}
	
	% Für Exponentenschreibweise ( Anwendung: 123\E{3} )
	\newcommand*{\E}[1]{\ensuremath{\cdot 10^{#1}}}
	
	\newcommand*{\eexp}[1]{\ensuremath{\mathrm{e}^{#1}}}
	\newcommand*{\iu}{\ensuremath{\mathrm{j}}}
	
	\newcommand*{\todots}{\ensuremath{,\,\hdots,\,}}

% Makros für Formeln
% ==================

	% Definition für Vektor und Matizen
	\newcommand*{\mat}[1]{{\ensuremath{\boldsymbol{\mathrm{#1}}}}}
	\newcommand*{\ma}[1]{{\ensuremath{\boldsymbol{\mathrm{#1}}}}}
	\newcommand*{\mas}[1]{\ensuremath{\boldsymbol{#1}}}
	\newcommand*{\ve}[1]{\ensuremath{\boldsymbol{#1}}}
	\newcommand*{\ves}[1]{\ensuremath{\boldsymbol{\mathrm{#1}}}}
	
	\newcommand*{\AP}{\ensuremath{\mathrm{AP}}}
	\newcommand*{\doti}{\ensuremath{(i)^\cdot}}
	
	\newcommand*{\inprod}[2]{\ensuremath{\langle #1,\,#2 \rangle}}
	
	\newcommand*{\ul}[1]{\underline{#1}}
	
	% gerades "d" (z.B. für Integral)
	\newcommand*{\ud}{\ensuremath{\mathrm{d}}}
	
	% normaler Text in Formeln
	\newcommand*{\tn}[1]{\textnormal{#1}}
	
	% nicht-kursive Schrift in Formeln
	\newcommand*{\mrm}[1]{\ensuremath{\mathrm{#1}}}
	
	% gerades "T" für Transponiert
	\newcommand*{\transp}{\ensuremath{\mathrm{T}}}
	
	% gerades "rg"
	\newcommand*{\rang}{\ensuremath{\operatorname{rg}}}
	
	% Für geklammerte Ausdrücke mit Index (Subscript)
	% (einmal mit kursiven Index, einmal mit geradem Index)
	\newcommand*{\grpsb}[2]{\ensuremath{\left(#1\right)_{#2}}}
	\newcommand*{\grprsb}[2]{\ensuremath{\left(#1\right)_{\mathrm{#2}}}}
	
	% Ableitungen und Integrale
		% "normale" Ableitung (mit geraden "d"s)
		\newcommand*{\normd}[2]{\ensuremath{\frac{\mathrm{d}#1}{\mathrm{d}#2}}}
		\newcommand*{\normdat}[3]{\ensuremath{\left.\frac{\mathrm{d} #1}{\mathrm{d} #2}\right|_{#3}}}
		
		% Materielle Ableitung
		\newcommand*{\matd}[2]{\ensuremath{\frac{\mathrm{D} #1}{\mathrm{D} #2}}}
		\newcommand*{\matdat}[3]{\ensuremath{\left.\frac{\mathrm{D} #1}{\mathrm{D} #2}\right|_{#3}}}
		
		% Partielle Ableitung
		\newcommand*{\partiald}[2]{\ensuremath{\frac{\partial #1}{\partial #2}}}
		\newcommand*{\partialdat}[3]{\ensuremath{\left.\frac{\partial #1}{\partial #2}\right|_{#3}}}
	
	
	% Transformationen
	\newcommand*{\FT}[1]{\ensuremath{\mathfrak{F}\left\{#1\right\}}}
	\newcommand*{\FTabs}[1]{\ensuremath{\left|\mathfrak{F}\left\{#1\right\}\right|}}
	\newcommand*{\IFT}[1]{\ensuremath{\mathfrak{F}^{-1}\left\{#1\right\}}}
	\newcommand*{\DFT}[1]{\ensuremath{\mathrm{DFT}\left\{#1\right\}}}
	\newcommand*{\DFTabs}[1]{\ensuremath{\left|\mathrm{DFT}\left\{#1\right\}\right|}}
	\newcommand*{\Laplace}[1]{\ensuremath{\mathfrak{L}\left(#1\right)}}
	\newcommand*{\InvLaplace}[1]{\ensuremath{\mathfrak{L^{-1}}\left(#1\right)}}
	\newcommand*{\invtrans}{\ensuremath{\quad\bullet\!\!-\!\!\!-\!\!\circ\quad}}
	\newcommand*{\trans}{\ensuremath{\quad\circ\!\!-\!\!\!-\!\!\bullet\quad}}
	
	
	\newcommand*{\mlfct}[1]{\texttt{#1}}
	\newcommand*{\mlvar}[1]{\texttt{#1}}
	
	
	% Manche textcomp-Zeichen funktionieren mit dem TU-Design nicht, diese können dann mit diesem
	% Befehl gesetzt werden.
	\newcommand*{\textcompstdfont}[1]{{\fontfamily{cmr} \fontseries{m} \fontshape{n} \selectfont #1}}

% Makros für Variablen
% ================================
\newcommand*{\amax}{\ensuremath{a_{\textrm{max}}}}
\newcommand*{\amin}{\ensuremath{a_{\textrm{min}}}}
\newcommand*{\jmax}{\ensuremath{j_{\textrm{max}}}}
\newcommand*{\jmin}{\ensuremath{j_{\textrm{min}}}}