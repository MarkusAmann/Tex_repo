%\documentclass[
%	paper=a4,
%	ngerman,
%	accentcolor=2c, % Akzentfarbe FG rtm
%	type=announcement,
%	marginpar=false,
%	fleqn,
%	class=report,
%	fontsize=11pt
%	]{tudaposter}
%
%%%%%%%%%%%%%%%%%%%%
%%Sprachanpassung & Verbesserte Trennregeln
%%%%%%%%%%%%%%%%%%%%
%\usepackage[english, main=ngerman]{babel}
%\usepackage[autostyle]{csquotes}% Anführungszeichen vereinfacht
%\usepackage{microtype}
%
%\usepackage{amsmath,amssymb}
%
%\usepackage{url}
%
%%%%%%%%%%%%%%%%%%%%
%%Literaturverzeichnis
%%%%%%%%%%%%%%%%%%%%
%\catcode30=12  % Workaround for current biblatex problem (https://tex.stackexchange.com/questions/564990/error-after-miktex-reinstall-text-line-contains-an-invalid-character)
%\usepackage[backend=biber,style=ieee,bibencoding=utf8]{biblatex}
%%\addbibresource{model_fidelity.bib}
%
%\renewcommand*{\bibfont}{\small}
%\newcommand{\un}[1]{\, \text{#1}}
%\newcommand{\ve}[1]{\mathbf{#1}}
%\newcommand{\veG}[1]{\text{\boldmath{$#1$}}}
%
%\begin{document}
%
%\title{Trajektorienplanung als dynamisches Optimalsteuerungsproblem und Interpretation des Lösungsraums für Fahrkomfort im automatisierten Fahren \vspace{0.2cm}}
%\subtitle{Aufgabenstellung zur Masterthesis}
%\titleinfo{Betreuer: Alexander Steinke, M.Sc.\\ \today}
%\addTitleBox{\includegraphics[width=\linewidth]{rtm_mit_schrift}}
%
%\maketitle
%%\section*{Aufgabenbeschreibung}
%
%%Zum numerischen Lösen Dynamischer Optimierungsprobleme können direkte und indirekt
%
%%Ein häufiger Ansatz zum Lösen dynamische Optimierungsprobleme ist die 
%
%%Dynamische Optimierungsprobleme lassen sich analytisch (Dynamische Optimierung) oder numerisch (Statische Optimierung) lösen. Während in der statischen Optimierung

Die allgemeine Struktur eines Optimalsteuerungsproblems lautet
%
\begin{equation*} \begin{aligned}
		\min\limits_{\ve{u}(\cdot)} \ \ \ & J(\ve{u})=V(\ve{x}(t_\text{f}),t_\text{f}) + \int_{t_0}^{t_\text{f}}l(\ve{x}(t),\ve{u}(t),t)\text{d}t \\
		\text{u.B.v.} \ \ \ & \dot{\ve{x}}=\ve{f}(\ve{x},\ve{u},t), \ \ve{x}(t_0)=\ve{x}_0\\ 
		& \mathbf{g}(\ve{x}(t_\text{f}),t_\text{f}) = \ve{0} \\
		& \mathbf{h}(\ve{x}(t),\ve{u}(t),t) \leq \ve{0}.
\end{aligned} \end{equation*}
%
Häufig werden dynamische Optimierungsprobleme (OP) mithilfe direkter Verfahren numerisch gelöst. 
In diesem Fall wird das dynamische in ein statisches OP überführt und der endliche Lösungsvektor $\ve{u}_\text{opt} \in \mathcal{U} \subseteq \mathbb{R}^m$ berechnet. 
Direkte Verfahren haben den Vorteil, dass Zustandsbeschränkungen leichter berücksichtigt werden können und der Konvergenzbereich größer ist. 
Indirekte Verfahren hingegen liefern eine Einsicht in die Struktur der optimalen Lösung.

In der dynamischen Optimierung hingegen werden Funktionen $\ve{u}(t)$ einer unabhängigen Variable $t$ gesucht. 
Mithilfe der Variationsrechnung können Optimalitätsbedingungen hergeleitet werden, die ein Randwertproblem formulieren. 
Die Lösung dieses Randwertproblems liefert in der Folge die optimale Steuertrajektorie $\ve{u}_\text{opt}(t)$. 
Jedoch ist das Lösen des Randwertproblems für nichtlineare Systeme häufig schwierig, weswegen auf numerische Verfahren zurückgegriffen werden muss.

Um im automatisierten Fahren (AD) gezielt Fahrprofile mittels einer optimalen Trajektorienplanung erstellen zu können, ist ein tieferes Verständnis der optimalen Lösung erforderlich. 
Mit einem direkten Verfahren ist die Interpretation der Lösung schwierig, da die Lösung aus reinen Zahlenwerten besteht. 
Zwar könnte ein gewünschtes Fahrverhalten in definierten Fahraufgaben durch zusätzliche Terme im Gütemaß und Anpassung der Gewichte erzielt werden, jedoch ist eine Übertragung auf andere Fahraufgaben fraglich. 
Eine parametrierte Steuertrajektorie $\ve{u}_\text{opt}(t)$ würde die Interpretation erheblich vereinfachen.


Ziele dieser Arbeit sind folgende Punkte:
\begin{enumerate}
	\item Übersicht für Lösungsverfahren dynamischer OPs und Einordnung des AD-Planungsproblems
	\item Erstellen von dynamischen OPs, die durch Variationsrechnung lösbar sind und Lösen dieser
	\begin{enumerate}
		\item Geeignete Fahrszenarien (z.B. Geradeausfahrt, Kurve, Geraden-Kurven-Kombination, Spurwechsel)
		\item Geeignete Komfortmerkmale (z.B. Beschleunigung, Ruck, Fahrdauer)
		\item Berücksichtigung von Begrenzungen
	\end{enumerate}
	\item Interpretation der Lösungenstrajektorien hinsichtlich des dargestellten Funktionenraums 
	\item Einordnung im Kontext Fahrkomfort
\end{enumerate}

Die Ergebnisse sind geeignet zu visualisieren und zu dokumentieren. Die aktuelle Fassung der Richtlinien zur Anfertigung von Abschlussarbeiten ist zu beachten.


%%Technische Universität Darmstadt \\
%%Institut für Automatisierungstechnik und Mechatronik \\
%%Fachgebiet Regelungstechnik und Mechatronik \\
%%Prof. Dr.-Ing. Ulrich Konigorski
%
%%\vspace*{\fill} {\tiny
%%\printbibliography[heading=none]}
%
%\end{document}
