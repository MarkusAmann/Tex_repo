\documentclass[
	ngerman,
	ruledheaders=section,%Ebene bis zu der die Überschriften mit Linien abgetrennt werden, vgl. DEMO-TUDaPub
	class=report,% Basisdokumentenklasse. Wählt die Korrespondierende KOMA-Script Klasse
	thesis={type=master},% Dokumententyp Thesis
	accentcolor=2c, % Akzentfarbe FG rtm
	custommargins=false,
	marginpar=false,
	BCOR=10mm, % Bindekorrektur (etwas größere mittlere Ränder)
	parskip=half-, % Absatzkennzeichnung durch Abstand vgl. KOMA-Sript
	fontsize=11pt, % Basisschriftgröße laut Corporate Design ist mit 9pt häufig zu klein
  	twoside,
  	numbers=noendperiod,
  	fleqn,
  	captions=oneline, % Einzeilige Captions zentrieren
	captions=tableheading, % Für besseren Abstand bei TabellenÜBERschriften
  	IMRAD=false,
]{tudapub}


%%%%%%%%%%%%%%%%%%%
%Pakete einbinden
%%%%%%%%%%%%%%%%%%%
%%%%%%%%%%%%%%%%%%%
% Mathe-Umgebungen und Symbole
%%%%%%%%%%%%%%%%%%%
\usepackage{amsmath}

%%%%%%%%%%%%%%%%%%%
% TikZ für Blockschaltbilder und Skizzen
%%%%%%%%%%%%%%%%%%%
\usepackage{tikz}				% Zum Erzeugen von Bildern mit TikZ
\input{Bilder/TikZ_BSBnormal.tex} % rtm-eigene Bibliothek für Blockschaltbilder

%%%%%%%%%%%%%%%%%%%
% Plotten direkt in LaTeX
%%%%%%%%%%%%%%%%%%%
\usepackage{pgfplots}

%%%%%%%%%%%%%%%%%%%
%Sprachanpassung & Verbesserte Trennregeln
%%%%%%%%%%%%%%%%%%%
\usepackage[english, main=ngerman]{babel}
\usepackage[autostyle]{csquotes}% Anführungszeichen vereinfacht
\usepackage{microtype}

%%%%%%%%%%%%%%%%%%%
%Meine Makros einbinden
%%%%%%%%%%%%%%%%%%%
\newcommand{\eqp}{\ensuremath{\, \, \, .}}
\newcommand{\M}[1]{\textbf{#1}} %Matrix  M
\newcommand{\Mr}[1]{\textbf{#1}_\rho} %Matrix mit tiefgestelltem rho, M_rho
\newcommand{\Mtil}[1]{\tilde{\textbf{#1}}} %Matrix mit Tilde
\newcommand{\Mtilt}[2]{\tilde{\textbf{#1}}_{#2}} %Matrix mit Tilde mit tiefgestelltem arg2
\DeclareRobustCommand{\Mt}[2]{\textbf{#1}_{#2}} %Matrix M mit tiefgestelltem arg2


\DeclareRobustCommand{\w}[1]{\underline{#1}} %Vektor arg1 unterstrichen 
\DeclareRobustCommand{\wt}[2]{\underline{#1}_{#2}} %Vektor unterstrichen w mit tiefgestelltem arg2
\DeclareRobustCommand{\wr}[1]{\underline{#1}_{\rho}} %Vektor unterstrichen mit rho, w_rho

% Textbausteine
% =============
	% Produktnamen
	\newcommand*{\Matlab}{\textsc{Matlab}}
	\newcommand*{\Matlabreg}{\textsc{Matlab}\textsuperscript{\tiny \textregistered}}
	\newcommand*{\MatSim}{\textsc{Matlab/Simulink}}
	\newcommand*{\Simulink}{\textsc{Simulink}}
	\newcommand*{\Simulinkreg}{\textsc{Simulink}\textsuperscript{\tiny \textregistered}}

	% Das Makro |\name|\marg{person} formatiert einen Personennamen bspw. eines Erfinders oder Entdeckers gemäß |\name{Euler}| \arrow\ \name{Euler}.
	\newcommand*{\name}[1]{\textsc{#1}}
	\newcommand*{\GenderPl}[1]{{#1}\_innen}
	\newcommand*{\GendeSin}[1]{{#1}\_in}
% Makros für Einheiten, Exponenten
% ================================

	\newcommand*{\unit}[1]{\ensuremath{\mathrm{#1}}}
	
	% Wert mit Einheit (mit kleinem Leerzeichen dazwischen), aus Text- UND Math-Modus
	\newcommand*{\valunit}[2]{\ensuremath{#1\,\mrm{#2}}}
	
	
	% "°C", im Text- oder Mathe-Modus
	\newcommand*{\degC}{
		\ifmmode
		^\circ \mrm{C}%
		\else
		\textdegree C%
		\fi}
	
	\newcommand*{\degree}{
		\ifmmode
		^\circ%
		\else
		\textdegree%
		\fi}
	
	% Für Exponentenschreibweise ( Anwendung: 123\E{3} )
	\newcommand*{\E}[1]{\ensuremath{\cdot 10^{#1}}}
	
	\newcommand*{\eexp}[1]{\ensuremath{\mathrm{e}^{#1}}}
	\newcommand*{\iu}{\ensuremath{\mathrm{j}}}
	
	\newcommand*{\todots}{\ensuremath{,\,\hdots,\,}}

% Makros für Formeln
% ==================

	% Definition für Vektor und Matizen
	\newcommand*{\mat}[1]{{\ensuremath{\boldsymbol{\mathrm{#1}}}}}
	\newcommand*{\ma}[1]{{\ensuremath{\boldsymbol{\mathrm{#1}}}}}
	\newcommand*{\mas}[1]{\ensuremath{\boldsymbol{#1}}}
	\newcommand*{\ve}[1]{\ensuremath{\boldsymbol{#1}}}
	\newcommand*{\ves}[1]{\ensuremath{\boldsymbol{\mathrm{#1}}}}
	
	\newcommand*{\AP}{\ensuremath{\mathrm{AP}}}
	\newcommand*{\doti}{\ensuremath{(i)^\cdot}}
	
	\newcommand*{\inprod}[2]{\ensuremath{\langle #1,\,#2 \rangle}}
	
	\newcommand*{\ul}[1]{\underline{#1}}
	
	% gerades "d" (z.B. für Integral)
	\newcommand*{\ud}{\ensuremath{\mathrm{d}}}
	
	% normaler Text in Formeln
	\newcommand*{\tn}[1]{\textnormal{#1}}
	
	% nicht-kursive Schrift in Formeln
	\newcommand*{\mrm}[1]{\ensuremath{\mathrm{#1}}}
	
	% gerades "T" für Transponiert
	\newcommand*{\transp}{\ensuremath{\mathrm{T}}}
	
	% gerades "rg"
	\newcommand*{\rang}{\ensuremath{\operatorname{rg}}}
	
	% Für geklammerte Ausdrücke mit Index (Subscript)
	% (einmal mit kursiven Index, einmal mit geradem Index)
	\newcommand*{\grpsb}[2]{\ensuremath{\left(#1\right)_{#2}}}
	\newcommand*{\grprsb}[2]{\ensuremath{\left(#1\right)_{\mathrm{#2}}}}
	
	% Ableitungen und Integrale
		% "normale" Ableitung (mit geraden "d"s)
		\newcommand*{\normd}[2]{\ensuremath{\frac{\mathrm{d}#1}{\mathrm{d}#2}}}
		\newcommand*{\normdat}[3]{\ensuremath{\left.\frac{\mathrm{d} #1}{\mathrm{d} #2}\right|_{#3}}}
		
		% Materielle Ableitung
		\newcommand*{\matd}[2]{\ensuremath{\frac{\mathrm{D} #1}{\mathrm{D} #2}}}
		\newcommand*{\matdat}[3]{\ensuremath{\left.\frac{\mathrm{D} #1}{\mathrm{D} #2}\right|_{#3}}}
		
		% Partielle Ableitung
		\newcommand*{\partiald}[2]{\ensuremath{\frac{\partial #1}{\partial #2}}}
		\newcommand*{\partialdat}[3]{\ensuremath{\left.\frac{\partial #1}{\partial #2}\right|_{#3}}}
	
	
	% Transformationen
	\newcommand*{\FT}[1]{\ensuremath{\mathfrak{F}\left\{#1\right\}}}
	\newcommand*{\FTabs}[1]{\ensuremath{\left|\mathfrak{F}\left\{#1\right\}\right|}}
	\newcommand*{\IFT}[1]{\ensuremath{\mathfrak{F}^{-1}\left\{#1\right\}}}
	\newcommand*{\DFT}[1]{\ensuremath{\mathrm{DFT}\left\{#1\right\}}}
	\newcommand*{\DFTabs}[1]{\ensuremath{\left|\mathrm{DFT}\left\{#1\right\}\right|}}
	\newcommand*{\Laplace}[1]{\ensuremath{\mathfrak{L}\left(#1\right)}}
	\newcommand*{\InvLaplace}[1]{\ensuremath{\mathfrak{L^{-1}}\left(#1\right)}}
	\newcommand*{\invtrans}{\ensuremath{\quad\bullet\!\!-\!\!\!-\!\!\circ\quad}}
	\newcommand*{\trans}{\ensuremath{\quad\circ\!\!-\!\!\!-\!\!\bullet\quad}}
	
	
	\newcommand*{\mlfct}[1]{\texttt{#1}}
	\newcommand*{\mlvar}[1]{\texttt{#1}}
	
	
	% Manche textcomp-Zeichen funktionieren mit dem TU-Design nicht, diese können dann mit diesem
	% Befehl gesetzt werden.
	\newcommand*{\textcompstdfont}[1]{{\fontfamily{cmr} \fontseries{m} \fontshape{n} \selectfont #1}}

% Makros für Variablen
% ================================
\newcommand*{\amax}{\ensuremath{a_{\textrm{max}}}}
\newcommand*{\amin}{\ensuremath{a_{\textrm{min}}}}
\newcommand*{\jmax}{\ensuremath{j_{\textrm{max}}}}
\newcommand*{\jmin}{\ensuremath{j_{\textrm{min}}}}

%%%%%%%%%%%%%%%%%%%
%Literaturverzeichnis
%%%%%%%%%%%%%%%%%%%
\catcode30=12  % Workaround for current biblatex problem (https://tex.stackexchange.com/questions/564990/error-after-miktex-reinstall-text-line-contains-an-invalid-character)
\usepackage[backend=biber,style=ieee,bibencoding=utf8]{biblatex}
\addbibresource{./Bib/literature.bib}

% ======================================================
% Fix für Fehlermeldungen 'Undifined control sequence' und 'Missing number, treated as zero.' bei gestapelten Akzenten in der Mathe-Umgebung. (z.B. \dot{\hat{x}})
% Der Fehler tritt auf bei tuda-ci v3.10 (Feb. 2021) in Kombination mit dem amsmath package.
%
% BITTE PRÜFEN, OB DER FEHLER NOCH BESTEHT UND DEN WORKAROUND GGF. WIEDER ENTFERNEN
%
% workaround/solution by skillmon posted here
% https://tex.stackexchange.com/questions/482281/math-accent-symbol-over-parentheses-enclosing-accented-symbol-amsmath
\makeatletter
\protected\def\mathaccentV#1#2#3#4#5%
  {%
    \ifmmode
      \mathaccentV@do{#2}{#3}{#4}{#5}%
    \else
      \@xp\nonmatherr@\csname #1\endcsname
    \fi
  }
\def\mathaccentV@do#1#2#3#4%
  {%
    \global\let\macc@nucleus\@empty
    \mathaccent"\accentclass@#1#2#3{#4}\macc@nucleus
  }
\makeatother
% ======================================================

\makeglossaries


%\newacronym{FAS}{FAS}{Fahrerassistenzsysteme}
%\newacronym{ACC}{ACC}{Adaptive Cruise Control}
%\newacronym{FSRA}{FSRA}{Full Speed Range Adaptive Cruise Control}
%\newacronym{FSW}{FSW}{Fahrstreifenwechsel}
\newglossaryentry{utc}{name=utc,description={Coordinated Universal Time}}
\newglossaryentry{FAS}{name=FAS,description={Fahrerassistenzsystem/-e}}
\newglossaryentry{ACC}{name=ACC,description={Adaptive Cruise Control}}
\newglossaryentry{FSRA}{name=FSRA,description={Full Speed Range Adaptive Cruise Control}}
\newglossaryentry{FSW}{name=FSW,description={Fahrstreifenwechsel}}
\newglossaryentry{LKS}{name=LKS,description={Lane Keeping Support}}
\setcounter{secnumdepth}{3}
\setcounter{tocdepth}{3}
\begin{document}
% =================================================================================
% Header mit allen Metadaten etc. 
% =================================================================================
\Metadata{	% Metadaten im erzeugten PDF/A
	title=Masterthesis Markus Amann,
	author=Markus Amann,
	keywords={TU Darmstadt \sep Corporate Design \sep LaTeX},
}

\title{Trajektorienplanung als dynamisches Optimalsteuerungsproblem und Interpretation des Lösungsraums für Fahrkomfort im automatisierten Fahren}
\subtitle{\textmd{Studiengang Elektrotechnik und Informationstechnik}}
\author[Markus Amann]{Markus Amann}
\reviewer*[Erstprüfer,Betreuer]{Prof. Dr.-Ing. Ulrich Konigorski \and Wi Mi M.Sc. Alexander Steinke} % erst ab Version v3.06 (2020-10-20) des tuda-ci-Pakets möglich

\department{etit}

\addTitleBox{\includegraphics[width=\linewidth]{Bilder/rtm_mit_schrift}}

%\date{\today}
\submissiondate{4. Juli 2022}

\lowertitleback{
	Technische Universität Darmstadt\\
	Institut für Automatisierungstechnik und Mechatronik\\
	Fachgebiet Regelungstechnik und Mechatronik\\
	Prof. Dr.-Ing. Ulrich Konigorski\\
}

\frontmatter
\maketitle

\clearpage
\setcounter{page}{1}
\section*{Aufgabenstellung}

%\documentclass[
%	paper=a4,
%	ngerman,
%	accentcolor=2c, % Akzentfarbe FG rtm
%	type=announcement,
%	marginpar=false,
%	fleqn,
%	class=report,
%	fontsize=11pt
%	]{tudaposter}
%
%%%%%%%%%%%%%%%%%%%%
%%Sprachanpassung & Verbesserte Trennregeln
%%%%%%%%%%%%%%%%%%%%
%\usepackage[english, main=ngerman]{babel}
%\usepackage[autostyle]{csquotes}% Anführungszeichen vereinfacht
%\usepackage{microtype}
%
%\usepackage{amsmath,amssymb}
%
%\usepackage{url}
%
%%%%%%%%%%%%%%%%%%%%
%%Literaturverzeichnis
%%%%%%%%%%%%%%%%%%%%
%\catcode30=12  % Workaround for current biblatex problem (https://tex.stackexchange.com/questions/564990/error-after-miktex-reinstall-text-line-contains-an-invalid-character)
%\usepackage[backend=biber,style=ieee,bibencoding=utf8]{biblatex}
%%\addbibresource{model_fidelity.bib}
%
%\renewcommand*{\bibfont}{\small}
%\newcommand{\un}[1]{\, \text{#1}}
%\newcommand{\ve}[1]{\mathbf{#1}}
%\newcommand{\veG}[1]{\text{\boldmath{$#1$}}}
%
%\begin{document}
%
%\title{Trajektorienplanung als dynamisches Optimalsteuerungsproblem und Interpretation des Lösungsraums für Fahrkomfort im automatisierten Fahren \vspace{0.2cm}}
%\subtitle{Aufgabenstellung zur Masterthesis}
%\titleinfo{Betreuer: Alexander Steinke, M.Sc.\\ \today}
%\addTitleBox{\includegraphics[width=\linewidth]{rtm_mit_schrift}}
%
%\maketitle
%%\section*{Aufgabenbeschreibung}
%
%%Zum numerischen Lösen Dynamischer Optimierungsprobleme können direkte und indirekt
%
%%Ein häufiger Ansatz zum Lösen dynamische Optimierungsprobleme ist die 
%
%%Dynamische Optimierungsprobleme lassen sich analytisch (Dynamische Optimierung) oder numerisch (Statische Optimierung) lösen. Während in der statischen Optimierung

Die allgemeine Struktur eines Optimalsteuerungsproblems lautet
%
\begin{equation*} \begin{aligned}
		\min\limits_{\ve{u}(\cdot)} \ \ \ & J(\ve{u})=V(\ve{x}(t_\text{f}),t_\text{f}) + \int_{t_0}^{t_\text{f}}l(\ve{x}(t),\ve{u}(t),t)\text{d}t \\
		\text{u.B.v.} \ \ \ & \dot{\ve{x}}=\ve{f}(\ve{x},\ve{u},t), \ \ve{x}(t_0)=\ve{x}_0\\ 
		& \mathbf{g}(\ve{x}(t_\text{f}),t_\text{f}) = \ve{0} \\
		& \mathbf{h}(\ve{x}(t),\ve{u}(t),t) \leq \ve{0}.
\end{aligned} \end{equation*}
%
Häufig werden dynamische Optimierungsprobleme (OP) mithilfe direkter Verfahren numerisch gelöst. 
In diesem Fall wird das dynamische in ein statisches OP überführt und der endliche Lösungsvektor $\ve{u}_\text{opt} \in \mathcal{U} \subseteq \mathbb{R}^m$ berechnet. 
Direkte Verfahren haben den Vorteil, dass Zustandsbeschränkungen leichter berücksichtigt werden können und der Konvergenzbereich größer ist. 
Indirekte Verfahren hingegen liefern eine Einsicht in die Struktur der optimalen Lösung.

In der dynamischen Optimierung hingegen werden Funktionen $\ve{u}(t)$ einer unabhängigen Variable $t$ gesucht. 
Mithilfe der Variationsrechnung können Optimalitätsbedingungen hergeleitet werden, die ein Randwertproblem formulieren. 
Die Lösung dieses Randwertproblems liefert in der Folge die optimale Steuertrajektorie $\ve{u}_\text{opt}(t)$. 
Jedoch ist das Lösen des Randwertproblems für nichtlineare Systeme häufig schwierig, weswegen auf numerische Verfahren zurückgegriffen werden muss.

Um im automatisierten Fahren (AD) gezielt Fahrprofile mittels einer optimalen Trajektorienplanung erstellen zu können, ist ein tieferes Verständnis der optimalen Lösung erforderlich. 
Mit einem direkten Verfahren ist die Interpretation der Lösung schwierig, da die Lösung aus reinen Zahlenwerten besteht. 
Zwar könnte ein gewünschtes Fahrverhalten in definierten Fahraufgaben durch zusätzliche Terme im Gütemaß und Anpassung der Gewichte erzielt werden, jedoch ist eine Übertragung auf andere Fahraufgaben fraglich. 
Eine parametrierte Steuertrajektorie $\ve{u}_\text{opt}(t)$ würde die Interpretation erheblich vereinfachen.


Ziele dieser Arbeit sind folgende Punkte:
\begin{enumerate}
	\item Übersicht für Lösungsverfahren dynamischer OPs und Einordnung des AD-Planungsproblems
	\item Erstellen von dynamischen OPs, die durch Variationsrechnung lösbar sind und Lösen dieser
	\begin{enumerate}
		\item Geeignete Fahrszenarien (z.B. Geradeausfahrt, Kurve, Geraden-Kurven-Kombination, Spurwechsel)
		\item Geeignete Komfortmerkmale (z.B. Beschleunigung, Ruck, Fahrdauer)
		\item Berücksichtigung von Begrenzungen
	\end{enumerate}
	\item Interpretation der Lösungenstrajektorien hinsichtlich des dargestellten Funktionenraums 
	\item Einordnung im Kontext Fahrkomfort
\end{enumerate}

Die Ergebnisse sind geeignet zu visualisieren und zu dokumentieren. Die aktuelle Fassung der Richtlinien zur Anfertigung von Abschlussarbeiten ist zu beachten.


%%Technische Universität Darmstadt \\
%%Institut für Automatisierungstechnik und Mechatronik \\
%%Fachgebiet Regelungstechnik und Mechatronik \\
%%Prof. Dr.-Ing. Ulrich Konigorski
%
%%\vspace*{\fill} {\tiny
%%\printbibliography[heading=none]}
%
%\end{document}


\vspace{0.5cm}
\begin{tabular}{ll}
	Beginn: 	& 03. Januar 2022 \\
	Ende: 		& 04. Juli 2022 \\
	Seminar: 	& ------ \\
\end{tabular}
\clearpage

\affidavit

\section*{Kurzfassung}
Zusammenfassung entsprechend der Dokumentensprache. In diesem Fall Deutsch.
\section*{Abstract}
Additional abstract in English.	

\cleardoublepage

\tableofcontents
\cleardoublepage

\mainmatter


% =================================================================================
% Kapitel und Sections etc.
% =================================================================================
\chapter{Einleitung}\label{cha:Einleitung}
In der Schnelllebigkeit der heutigen Gesellschaft stellt das \gls{AD} ein Thema dar, welches seit Jahren einen festen Platz im gesellschaftlichen Diskurs in unterschiedlichen Bereichen einnimmt. Neben der technischen Weiterentwicklung werden auch rechtliche, ethische, ökonomische und ökologische Aspekte intensiv hinsichtlich möglicher Chancen und eventueller Risiken diskutiert. Gerade in der gegenwärtigen Debatte um den Klimawandel sehen mehrere Studien in der Weiterentwicklung autonomer Fahrzeuge das Potential zur Reduktion von klimaschädlichen Emissionen durch den Verkehrssektor \cite{Brost.2020,MichaelKrail.2019,Lee.2019}. Durch eine energieoptimierte Fahrweise und die effiziente Nutzung automatisierter Fahrzeuge, können enorme Energie- und Treibhausgaseinsparungen erzielt werden \cite{MichaelKrail.2019}. Damit das Potential zur Energie- und Emissionsreduktion jedoch ausgeschöpft werden kann, muss das \gls{AD} im richtigen Rahmen eingesetzt werden. Sekundäre Effekte wie erhöhtes Verkehrsaufkommen durch die zusätzliche Attraktivität privater Einzelnutzung automatisierter Fahrzeuge oder eine hohe Zahl von Leerfahrten können auch negativen Einfluss auf die Energiebilanz des Verkehrs- und Transportsektors haben \cite{MichaelKrail.2019,AgoraVerkehrswende.082020}. Solche Effekte gilt es unbedingt zu vermeiden, weshalb politische und gesellschaftliche Rahmenbedingungen mit Fokus auf die nachhaltige Nutzung für die Einführung automatisierter Fahrzeuge unerlässlich sind -- der technologische Fortschritt alleine wird darüber nicht entscheiden können \cite{AgoraVerkehrswende.082020}. Weiteres Verbesserungspotential bietet das \gls{AD} zudem im Bereich der Verkehrssicherheit. Trotz all der Sensorik und der Vielzahl an \gls{FAS}, die heutzutage in Fahrzeugen verbaut werden, ist der Mensch als fahrzeugführende Person immer noch der Hauptgrund für Unfälle im Straßenverkehr. Während insgesamt \valunit{88}{\%} aller Unfälle ihre Ursache in menschlichem Fehlverhalten haben, beträgt der Einfluss technischer Mängel gerade Mal \valunit{1}{\%}. Dies zeigen Zahlen des statistischen Bundesamts aus dem Jahre 2017 \cite{StatistischesBundesamt.2017}. Daher bietet das \gls{AD} das Potential zur Steigerung an Verkehrssicherheit und Reduktion der Unfallzahlen, sowohl im innerstädtischen Bereich, als auch auf Autobahnen. 

Neben dem Sicherheitsaspekt bei der Entwicklung automatisierter Fahrzeuge, kommt dem Fahrkomfort eine besondere Bedeutung zu. Gegenüber der Personen geführten Fahrt, steigt die Notwendigkeit einen hohen Fahrkomfort zu erreichen beim \gls{AD} sogar noch an. Das gesteigerte Komfortbedürfnis kann damit begründet werden, dass beim \gls{AD} zunehmend fahrfremde Tätigkeiten ausgeübt werden können. Unter fahrfremden Tätigkeiten werden solche Aktivitäten verstanden, die über die Navigation, das Führen und die Stabilisierung des Fahrzeugs sowie die Überwachung des Umfeld und des umgebenden Verkehrs hinausgehen \cite{Festner.2019}. Als Beispiel können Tätigkeiten wie Essen und Trinken, das Lesen von Zeitschriften oder Büchern, das Arbeiten und Beantworten von Nachrichten am Laptop oder Smartphone oder auch das Schlafen genannt werden. Gemäß der SAE-Klassifizierung \cite{SAETaxonomy.2018} unterschiedlicher Stufen von Automatisierung im Fahrzeug, ist spätestens ab Level 3 davon auszugehen, dass zumindest zeitweise fahrfremde Tätigkeiten ausgeübt werden können \cite{Festner.2019}. Ab diesem Level muss die fahrzeugführende Person zwar noch als Rückfallebene jederzeit dazu bereit sein, die Kontrolle über das Fahrzeug zu übernehmen. Im \glqq normalen\grqq~Anwendungsfall hingegen obliegen die Fahrzeugführung sowie die Verkehrs- und Umfeldüberwachung dem System \cite{SAETaxonomy.2018}. Mögliche Einschränkungen, die bei Level 3 für die fahrfremden Tätigkeiten gelten, nehmen mit zunehmendem Grad an Automatisierung ab \cite{Festner.2019}. Das Ausüben von fahrfremden Tätigkeiten begünstigt dabei das Auftreten von Kinetose, auch bekannt als Reisekrankheit, was einerseits an einem Sensorkonflikt zwischen der visuellen Information des Auges und der vestibulären Sensorinformation des Gleichgewichtsorgans liegt \cite{Reason.1975,Golding.2006,Benson.2002}. Andererseits wird die Kontrolle über das Fahrzeug und dessen Verhalten mit steigender Automatisierung zunehmend an das System übergeben, wodurch Fahrzeugbewegungen von den Insassen nicht kontrolliert und dadurch weniger genau prädiziert werden können, was wiederum das Auftreten von Kinetose \cite{Rolnick.1991} begünstigt. Vor diesem Hintergrund ist die Berücksichtigung des Fahrkomforts bei der Planung optimaler Fahrzeugtrajektorien für die Weiterentwicklung des \gls{AD} unabdingbar. Wenngleich die Manöver- und Trajektorienplanung nur einen Teil der gesamten Systemkette eines automatisierten Fahrzeugs darstellt, ist dieser doch elementar für das Erzielen eines hohen Fahrkomforts, da in diesem Block maßgeblich die optimalen Zustands- und Stellgrößentrajektorien geplant werden. 

In der Literatur existieren verschiedene Herangehensweisen, die Trajektorienplanung für das \gls{AD} umzusetzen. Alle Ansätze haben gemein, dass das Planungsproblem zunächst mithilfe mathematischer Formulierungen durch ein \gls{OP} ausgedrückt wird. Dieses besteht aus einer Gütefunktion oder einem Gütefunktional, welches der Minimierung von unterschiedlichen Gütekriterien dient, sowie Nebenbedingungen, durch die die Beschreibung der Systemdynamik, Randwerte und Beschränkungsanforderungen an die Zustände und Eingangsgrößen berücksichtigt werden können. Davon ausgehend bieten sich mehrere Möglichkeiten zur Lösung des \gls{OP}. In \cite{shi} und \cite{guo} wird eine \gls{MPC} für die Planung optimaler Fahrzeugtrajektorien verwendet. Bei der Anwendung von \gls{MPC} wird das Optimalsteuerungsproblem durch Rückführung des Fahrzeugzustands wiederholt gelöst. Dabei wird ein (meist lineares) Modell verwendet, welches dazu dient, das Fahrzeugverhalten über einen gewissen Prädiktionshorizont vorherzusagen. Diese Herangehensweise kann sowohl zur Planung der optimalen Fahrzeugtrajektorien, als auch zur Fahrzeugstabilisierung verwendet werden. Vorteil von \gls{MPC} ist, dass Zustandsbeschränkungen zur Kollisionsvermeidung, oder um zu verhindern, dass das Fahrzeug Fahrbahnbegrenzungen verletzt, vergleichsweise einfach berücksichtigt werden können. Ein entscheidender Nachteil beim Einsatz von \gls{MPC} liegt darin, dass die zyklische Lösung des Optimalsteuerungsproblems zu einem enormen Rechenaufwand führen kann. Zudem wird nach jedem Planungsschritt immer nur der erste Wert der optimalen Lösung tatsächlich verwendet, sodass je nach Länge des Prädiktionshorizonts nicht nur der Rechenaufwand stark ansteigt, sondern gleichzeitig ein Großteil der erhaltenen Lösung effektiv gar nicht verwendet wird.  

In \cite{Christ.2021} wird das \gls{OP} zur Erzeugung zeitoptimaler Trajektorien für ein Rennfahrzeug mithilfe eines direkten Kollokationsverfahrens gelöst, bei dem das \gls{OP} zunächst auf einem feinen, äquidistanten Gitter diskretisiert wird. Die Trajektorien der Stellgrößen werden anschließend auf jedem Teilintervall des Gitter als stückweise konstant angenommen, während die Trajektorien der Zustände auf jedem Teilintervall durch Polynome 3. Ordnung approximiert werden. Die Übereinstimmung der optimalen Trajektorien mit den Polynomen wird an sogenannten Kollokationspunkten überprüft. Zusätzlich werden Stetigkeitsbedingungen eingeführt, die einen kontinuierlichen Übergang zwischen den Polynomen an den Intervallgrenzen ermöglichen. Bei der Anwendung direkter Lösungsverfahren wird das \gls{OP} immer zunächst an zahlreichen Stützstellen diskretisiert und anschließend mithilfe eines numerischen Verfahrens gelöst. Vorteile dieser Vorgehensweise liegen darin, dass sich direkte Verfahren vielseitig und generisch einsetzen lassen und dabei eine hohe Robustheit besitzen. Zudem lassen sich Zustands- oder Pfadbegrenzungen leicht realisieren. Auf der anderen Seite ergeben sich durch die Diskretisierung Probleme mit einer großen Anzahl an Variablen, die sich im Rechenaufwand niederschlagen. 

Dementgegen stehen indirekte Lösungsverfahren, bei denen das \gls{OP} anstelle der Diskretisierung mithilfe der Variationsrechnung in ein \gls{ZPR} für ein System aus \gls{DGL} erster Ordnung überführt wird. Die Herangehensweise durch indirekte Verfahren bietet den Vorteil Lösungen sehr hoher Genauigkeit und zudem Einsicht in die Struktur der optimalen Trajektorien zu erhalten. Diese Vorgehensweise wird in \cite{Rathgeber.2016} gewählt. Dabei werden zunächst die Längs- und Querbewegung des Fahrzeugs getrennt betrachtet und unter Berücksichtigung gewählter Optimierungskriterien (Ruckänderung und Zeit) wird jeweils ein Gütefunktional mithilfe der Variationsrechnung minimiert, aus dem sich als Lösung die optimalen Stellgrößen- und Zustandstrajektorien für die Längs- und Querrichtung des Fahrzeugs bestimmen lassen. Diese Optimierung wird für mehrere Sets von Zielzuständen und Endzeitpunkten durchgeführt und alle Ergebnisse der Längsrichtung mit den Ergebnissen der Querrichtung kombiniert, sodass sich eine Schar von Trajektorien ergibt. Aus dieser Schar wird die beste Kombination im Sinne der Gütekriterien ausgewählt und als optimale Lösung betrachtet. Dabei werden die Trajektorien der Schar zusätzlich auf das Einhalten von Restriktionen wie Aktor- oder Fahrdynamikbegrenzungen überprüft und unzulässige Lösungen ausgeschlossen. Aus der getrennten Betrachtung von Quer- und Längsdynamik sowie der Optimierung bezüglich der Ruckänderung, resultiert ein Lösungsraum, der aus Polynomen besteht. Vorteil dieser Herangehensweise ist, dass sich damit ein echtzeitfähiger Planungsalgorithmus entwickeln lässt, wie \cite{Rathgeber.2016} zeigt. Allerdings unterliegt die Vorgehensweise der Annahme, dass Längs- und Querbewegung getrennt voneinander betrachtet werden können, wodurch die Ergebnisse weniger genau sind im Vergleich zur gekoppelten Betrachtung. 

Ein ähnliches Planungskonzept wird in \cite{Werling.2011} vorgestellt. Die Trajektorienplanung wird dabei als Teil einer neuartigen Bi-Level-Stabilisierung betrachtet, welche aus zwei Regelungen auf unterschiedlichen Ebenen besteht. Durch rekursive Trajektorienplanung in Abhängigkeit des Fahrzeugzustands werden die optimalen Steuergrößen geliefert, ohne dabei eine Regeldifferenz zu bilden. Diese Art der Vorsteuerung wird als High-Level-Stabilisierung bezeichnet und hat den Vorteil, dass auf impulsförmige Störungen durch Planung einer neuen Trajektorie reagiert wird, bei der weiterhin die Minimierung komfortrelevanter Kriterien im Fokus steht \cite{Werling.2011}. Auf der darunter liegenden Ebene wird die sogenannte Low-Level-Stabilisierung durchgeführt. Dabei werden die aus der Planung resultierenden optimalen Trajektorien mit dem Fahrzeugzustand bei einem für Regelungen üblichen Soll-/Istwert Vergleich zusammengeführt. Vorteil der unterlagerten Low-Level-Stabilisierung ist, dass permanente Störungen sowie Modellfehler ausgeglichen werden können \cite{Werling.2011}. Durch die Verknüpfung beider Stabilisierungsstrategien zur Bi-Level-Stabilisierung, lassen sich die Vorteile beider Einzelstrategien ausnutzen.   

Das \gls{AD}-Planungsproblem wird in der vorliegenden Arbeit unter Anwendung der Variationsrechnung mit besonderem Fokus auf den Fahrkomfort gelöst. Zunächst wird der Begriff Fahrkomfort erörtert und in den Kontext des \gls{AD} eingeordnet. Im Zuge dessen werden relevante Komfortkriterien mithilfe von Literaturquellen herausgearbeitet und diskutiert und es werden einige Richtwerte für die entsprechenden Komfortkriterien angegeben. Anschließend wird eine Einführung in die mathematischen Grundlagen der dynamischen Optimierung und speziell in die Variationsrechnung gegeben. Zudem werden einige Methoden als Erweiterung des \glqq klassischen\grqq~dynamischen \gls{OP} vorgestellt, die für die Lösung verschiedener Fahrszenarien benötigt werden. Als Abschluss dieses Kapitels werden mehrere indirekte Lösungsverfahren miteinander vergleichen und hinsichtlich ihrer Anwendbarkeit zur Lösung des \gls{AD}-Planungsproblems diskutiert. Danach wird ein Modell zur Beschreibung der Fahrzeugbewegung entlang einer vorgegebenen Referenzkurve in Frenet-Koordinaten hergeleitet. Schließlich werden einige selbst gewählte Fahrszenarien mithilfe der vorgestellten Methoden eingehend analysiert und unter dem Aspekt des Fahrkomforts interpretiert. Als Abschluss werden die gesammelten Ergebnisse zusammengefasst und ein Ausblick für weitere Forschung zu diesem Thema gegeben. 

\textbf{Hinweis:} In dieser Arbeit werden die Begriffe Stellgröße, Eingangsgröße und Steuergröße synonym verwendet und die abwechselnde Verwendung dieser dient ausschließlich der besseren Lesbarkeit. 


%die idealerweise eins zu eins von der Fahrzeugregelung umgesetzt werden. Das Blockschaltbild in Abbildung \ref{fig:Blockschaltbild_Fahrzeug} soll dazu die gesamte Systemkette von der Streckenplanung auf der Navigationsebene über die eigentliche Trajektorien- und Manöverplanung auf der Bahnführungsebene über die Fahrzeugregelung auf der Stabilisierungsebene bis hin zur Fahrzeugdynamik und äußeren Einflüssen durch die Umwelt und den umliegenden Verkehr schematisch darstellen. Die rote Umrandung hebt den für die Trajektorienplanung und die Generierung der optimalen Zustands- und Stellgrößentrajektorien $\ve{x}^*$ und $\ve{u}^*$ verantwortlichen Teil hervor. Das Modell ist dabei in seiner Struktur und in der Verwendung der Begrifflichkeiten an das Drei-Ebenen-Modell nach Donges angelehnt \cite{Donges}. 

%\begin{figure}[h]
%\centering
%\begin{tikzpicture}[auto, node distance=2cm,>=latex']
%\node [block, name=StrPlanung, align=left] {\textbf{Strategische Planung}\\ Navigation/Ortung};
%\node [block, above of=StrPlanung, name=Verkehr, node distance=2cm, align=left] {\textbf{Straßennetz}\\ Fahrtroute};
%\node [block, right of=StrPlanung, name=TaktPlanung, node distance=4.5cm, align=left] {\textbf{Taktische Planung}\\ Manöver/Trajektorien};
%\node [block, above of=TaktPlanung, name=Umfelderfassung, node distance=2cm, align=left] {\textbf{Umfelderfassung}\\ Verkehrsgeschehen};
%\node [block, right of=TaktPlanung, name=Regelung, node distance=4cm, align=left] {\textbf{Regelung}};
%\node [block, right of=Regelung, name=Aktorik, node distance=2.5cm, align=left] {\textbf{Aktorik}};
%\node [block, right of=Aktorik, name=Fahrzeug, node distance=3cm, align=left] {\textbf{Fahrzeugdynamik}};
%\node [block, below of=Fahrzeug, name=Messung, node distance=2cm, align=left] {\textbf{Messung/Schätzung}};
%
%\draw [to] (StrPlanung) -- (TaktPlanung);
%\draw [to] (TaktPlanung) -- node[name=uopt] {$\ve{u}^*$} node[name=xopt, below] {$\ve{x}^*$} (Regelung);
%\draw [to] (Regelung) -- node[name=u] {$\ve{u}$} (Aktorik);
%\draw [to] (Aktorik) -- (Fahrzeug);
%\draw [to] (Fahrzeug) -- node[name=x] {$\ve{x}$} (Messung);
%\draw [to] (Messung) -| (TaktPlanung);
%\draw [to] (Messung) -| node[below of=Regelung, node distance=0.3cm, name=xhat] {$\hat{\ve{x}}$} (Regelung);
%\draw [to] (Verkehr) -- (StrPlanung);
%\draw [to] (Umfelderfassung) -- (TaktPlanung);
%\draw[red,thick] ($(TaktPlanung.north west)+(-0.3,0.3)$)  rectangle ($(TaktPlanung.south east)+(0.8,-0.3)$);
%%\begin{pgfonlayer}{background}
%%\filldraw [line width=4mm,join=round,black!10] ($(Verkehr.north west)+(-0.1,0.6)$)  rectangle ($(Umfelderfassung.south east)+(0.1,-0.3)$);
%%\filldraw [line width=4mm,join=round,black!10] ($(StrPlanung.north west)+(-0.1,0.3)$)  rectangle ($(StrPlanung.south east)+(0.1,-0.6)$);
%%\filldraw [line width=4mm,join=round,black!10] ($(TaktPlanung.north west)+(-0.1,0.3)$)  rectangle ($(TaktPlanung.south east)+(0.1,-0.6)$);
%%\filldraw [line width=4mm,join=round,black!10] ($(Regelung.north west)+(-0.1,0.3)$)  rectangle ($(Regelung.south east)+(0.1,-0.6)$);
%%\filldraw [line width=4mm,join=round,black!10] ($(Aktorik.north west)+(-0.1,0.3)$)  rectangle ($(Messung.south east)+(0.1,-0.6)$);
%%\end{pgfonlayer}
%\end{tikzpicture}
%\caption{Blockschaltbild der gesamten Systemkette eines automatisierten Fahrzeugs in Anlehnung an das Drei-Ebenen-Modell nach Donges \cite{Donges}.}
%\label{fig:Blockschaltbild_Fahrzeug}
%\end{figure}


\chapter{Fahrkomfort und Einordnung in den Kontext Autonomes Fahren}\label{cha:Komfort}
In diesem Kapitel soll der Begriff ``Fahrkomfort'' in den Kontext des Autonomen Fahrens eingeordnet werden. Dazu wird zunächst die Bedeutung des Begriffs genauer betrachtet und die Verbindung zwischen Fahrkomfort und Sicherheitsempfinden dargelegt. Anschließend werden verschiedene Einflussfaktoren auf die Komfortwahrnehmung während des Fahrens thematisiert. Dabei wird das Phänomen der Kinetose genauer erklärt und speziell auf dessen Bedeutsamkeit für die Entwicklung autonomer Fahrzeuge eingegangen. Abschließend werden mithilfe von Literaturwerten Grenzwerte für die als relevant erarbeiteten Einflüsse untersucht. 

%\section{Begriffsdefinition Komfort}
%Während der Begriff ``Komfort'' im alltäglichen Sprachgebrauch eine sehr vielseitige Bedeutung hat und dabei sowohl zur Beschreibung positiver Empfindungen wie Wohlbehagen und Zufriedenheit, als auch in negierter Form zur Beschreibung negativer Gefühle wie Unwohlsein und Unzufriedenheit verwendet wird, wird bei der wissenschaftlichen Verwendung oftmals stärker differenziert. Es exisitieren verschiedene Modelle, die alle darauf abzielen, eine allgemeine Definition von Komfort zu liefern. Im Sinne der besseren Unterscheidbarkeit und damit einer genaueren Anwendung der entsprechenden Bedeutung, wird häufig zusätzlich der Begriff ``Diskomfort'' verwendet. In \cite{Zhang} wurden in einer Studie Faktoren für das Empfinden von Komfort und Diskomfort beim Sitzen identifiziert, wodurch sich die beiden Begriffe differenzierter verwenden lassen. Während Komfort durch Gefühle, die allgemein als positiv wahrgenommen werden, wie Entspanntheit, Ruhe und Zufriedenheit, klassifiziert wird, zeichnet sich Diskomfort vor allem durch negativ konnotierte Empfindungen, wie Müdigkeit und Rastlosigkeit, aber auch durch biometrische Faktoren, die spürbare Schmerzen verursachen, aus. Die Autoren leiten ein Modell her, in dem Komfort und Diskomfort zwei orthogonale Größen sind, die auch gleichzeitig erfahrbar sind. Während die Abwesenheit von Diskomfort nicht automatisch zur Zunahme von Komfort führt (dasselbe gilt auch anders herum), führt eine Zunahme von Diskomfort zwangsläufig zu einer Abnahme von Komfort. Wie Festner in \cite{Festner_diss} herausarbeitet, ist die Beseitigung von Diskomfort eine notwendige, jedoch nicht hinreichende Bedingung für das Empfinden von Komfort. Da jedoch unangenehme Situationen oftmals deutlicher explizit als solche wahrgenommen werden können, denn angenehme Situationen als angenehm, kann auch der erlebte Diskomfort als Kriterium für den fehlenden Komfort genutzt werden \cite{Festner}. 
%
%Ein weiteres Modell, welches breite Verwendung findet, ist die Komfortpyramide nach Krist \cite{Krist}. 
%\begin{figure}[h]
%	\centering
%	\includegraphics[scale=0.3]{./Bilder/komfortpyramide.png}
%	\label{fig:komfortpyramide}
%	\caption{Komfortpyramide aus \cite{Festner}, zitiert nach \cite{Krist}}
%\end{figure}
%
%In dieser Pyramide nimmt das Komfortlevel, ausgehend von sehr elementaren Faktoren wie Geruch und Lichteinflüssen, stetig zu. Um ein hohes Maß an Komfort zu erzielen, müssen notwendiger Weise zunächst die unteren, grundlegenden Bedingungen erfüllt sein. Zwar werden in diesem Modell Komfort und Diskomfort nicht so streng von einander getrennt, wie in \cite{Zhang}, allerdings ist die Bedingung, dass die Faktoren auf den unteren Ebenen zuerst erfüllt sein müssen damit auch auf den oberen Ebenen die Komfortbedingungen erfüllt werden können, vergleichbar mit der notwendigen Beseitigung von Diskomfort. Neben der qualitativen Beschreibung von Komfort, die alle Modelle gemein haben, unterscheiden sich die Ergebnisse, die unter Verwendung dieser Modelle erzielt werden, dennoch zum Teil stark, da die Komfortwahrnehmung sehr subjektiv ist und sich zwischen einzelnen Personen stark unterscheiden kann. Es lassen sich zwar einzelne Merkmale als komfortabel oder unkomfortabel klassifizieren, allerdings geht aus den Modellen nicht eindeutig hervor, wie stark die einzelnen Merkmale das persönliche Komfortempfinden beeinflussen. Zum Beispiel entwickelt jeder Mensch seine eigene Komfortpyramide, abhängig von ganz persönlichen Faktoren und Erfahrungen \cite{Krist}.
 
\section{Fahrkomfort im Kontext Autonomes Fahren}
Mit zunehmender Automatisierung von Fahraufgaben rückt auch der Fahrkomfort immer mehr in den Fokus. Es ist nicht davon auszugehen, dass FAS, die vom Fahrer als unkomfortabel empfunden werden, eine hohe Akzeptanz finden werden. \\
Damit sich überhaupt ein komfortables Gefühl während einer automatisierten Fahrt einstellen kann, muss beim Fahrer ein Gefühl von Sicherheit herrschen \cite{Festner.2019}. Nicht nur für unbeteiligte Dritte wie Fußgänger oder andere Autofahrer, sondern vor allem auch für den Fahrer spielt die Sicherheit bei autonomen Fahrzeugen daher eine zentrale Rolle. In \cite{Festner.2019} wird dabei der Unterschied zwischen subjektiver und objektiver Sicherheit dargelegt. Während das subjektive Sicherheitsempfinden lediglich wiedergibt, wie der Fahrer die Fahrt oder einzelne Situationen empfindet, lässt sich die objektive Sicherheit mithilfe von Größen wie reduzierten Unfallzahlen quantifizieren. Allerdings kann das subjektive Sicherheitsempfinden durchaus von der objektiven Sicherheit abweichen. Insbesondere bei FAS, die nach der SAE-Klassifizierung \cite{SAE klassifizierung} Stufe 3 oder höher zugeordnet werden, und bei denen damit das Fahrzeug nicht nur die Längs- und Querführung, sondern zusätzlich auch die Umfeldüberwachung übernimmt, kommt der in \cite{Elbanhawi.2015} als ``\textit{loss of controllability}'' eingeführte Paradigmenwechsel von Fahrer zum Passagier zum Tragen. Dadurch, dass der Fahrer mehr und mehr die Fahrzeugführung und Überwachung aller Funktionen an das Fahrzeug abgibt, stellt sich möglicherweise ein Gefühl von Kontrollverlust ein, weshalb nicht nur die objektive Sicherheit neuartiger Systeme eine große Rolle bei der Bewertung der Systeme spielt, sondern auch das subjektive Sicherheitsempfinden, welches eng an den wahrgenommenen Komfort geknüpft ist. \\
Neben dem Sicherheitsempfinden gibt es weitere wichtige Faktoren, die den Fahrkomfort beeinflussen. Großen Einfluss auf das Komfortempfinden hat dabei der Fahrstil. In verschiedenen Studien wurden unterschiedliche Fahrstile identifiziert und in verschiedene Klassen unterteilt \cite{Abendroth, B.; Bruder, R in Handbuch Fahrerassistenzsysteme, Bellem, Murphey}\cite{Abendroth.2009}\cite{Bellem.2016}\cite{Murphey.30.03.200902.04.2009}. So variiert zwar die Anzahl und genaue Bezeichnung der einzelnen Klassen zwischen den verschiedenen Studien, jedoch erstreckt sich das Spektrum immer von ``langsam'' oder ``komfortorientiert'' bis hin zu ``dynamisch/sportlich'' oder sogar ``aggressiv''. In \cite{Lange}\Cite{Lange.2014} konnte gezeigt werden, dass der Wunsch nach höherem Komfort mit dem Grad der Automatisierung steigt. Ein Grund für diesen Zusammenhang ist unter anderem, dass das Bedürfnis nach Feedback durch die Straße über das Fahrzeug an den Fahrer mit zunehmender Automatisierung geringer wird. Daher liegt für autonome Fahrzeuge im Alltagsgebrauch ein komfortorientierter Fahrstil nahe. 

Um beurteilen zu können, ob ein Fahrstil als komfortabel oder unkomfortabel wahrgenommen wird, muss zunächst geklärt werden, welche Merkmale dabei besonders großen Einfluss auf das Komfortempfinden haben. In \cite{scherer} wurden in einer Fahrsimulatorstudie die Merkmale Sicherheitsabstand zum Vorderfahrzeug, Bremsen, Geschwindigkeit, Beschleunigung, Spurhalten, Lenken und Blinkernutzung als die am häufigsten genannten Komfortkriterien identifiziert, wobei die drei ersten Merkmale von über der Hälfte der Teilnehmer als relevant erachtet wurden.\\
Neben den hier identifizierten Komfortmerkmalen lässt sich auch die Reisezeit als weiteres Kriterium angeben. Bereits in \cite{oborne}\cite{Oborne.1978} wurde dargelegt, dass Passagiere, die einer längeren Reisezeit ausgesetzt sind, ein höheres Maß an Komfort benötigen, als bei einer kürzeren Reisezeit. Jeder Passagier geht daher einen subjektiven Kompromiss zwischen der erwarteten Reisezeit und dem akzeptierten Niveau an Diskomfort ein. Aufgrund dessen hängen die vom Fahrer als komfortabel empfundenen biomechanischen Werte auch von der Reisezeit ab \cite{Festner.2019}.

Eine besondere Gewichtigkeit bei der Trajektorienplanung bezüglich des Fahrkomforts kommt dem Beschleunigungsverhalten zu. In \cite{Bellem}\cite{Bellem.2016} konnte in zwei Studien für verschiedene Fahrmanöver nachgewiesen werden, dass sich Merkmale wie Längs- und Querbeschleunigung sowie der Fahrzeugruck gut dazu eignen, einen komfortablen Fahrstil von anderen Fahrstilen abzugrenzen. Laut \cite{elbanhawi}\cite{Elbanhawi.2015} ist der häufigste Ansatz zur Optimierung der Fahrzeugbewegungen, die auf den Passagier wirkenden Kräfte und Rucke zu minimieren. Dass bei der Betrachtung des Beschleunigungsverhaltens jedoch nicht nur die absoluten Werte der Beschleunigung relevant sind, zeigen die Untersuchungen in \cite{gianna}\cite{Gianna.1996}. Hier konnte festgestellt werden, dass der Mensch bei lateralen Bewegungen insbesondere gegenüber Änderungen der Beschleunigung sehr sensitiv reagiert und diese besonders gut wahrnimmt. In \cite{dovgan} konnte ein Algorithmus entwickelt werden, der komfortable Fahrstrategien liefert, wobei als zusätzliches Komfortmerkmal der Ruck verwendet wurde. Dieser sollte dazu im Sinne des Fahrkomforts so gering wie möglich sein.

\subsection{Kinetose und ihre Bedeutung für das Autonome Fahren}\label{sec:kinetose}
Kinetose, oftmals auch als Reise- oder Bewegungskrankheit bezeichnet (\textit{engl.} Motion Sickness), beschreibt das Phänomen, dass man sich bei Reisen - vor allem im Auto, Flugzeug oder auf einem Schiff - plötzlich unwohl fühlt. Die Symptome reichen dabei von Blässe und leichter Übelkeit, über Schwindel und Kopfschmerzen, bis hin zum Erbrechen \cite{money}\cite{Money.1970}. Es existieren mehrere Theorien für die genauen Ursachen von Kinetose \cite{money}\cite{Money.1970}. Eine anerkannte und weit verbreitete Theorie für das Auftreten von Kinetose liegt dabei in einem visuell-vestibulären Sensorkonflikt \cite{motion sickness golding, motion sickness reason, motion sickness benson}\cite{Golding.2006}\Cite{Reason.1975}. Dadurch, dass Sensorinformationen über Beschleunigungen, die über das Gleichgewichtsorgan aufgenommen werden, nicht zu den Informationen passen, die über den visuellen Informationskanal geliefert werden, können die zuvor genannten Symptome auftreten. Dabei können prinzipiell alle Menschen mit einem funktionsfähigen Vestibularapparat gleichermaßen von Kinetose betroffen sein \cite{motion sickness lackner}\cite{Lackner.2014}. 

Neben dem Einfluss des Sensorkonflikts auf die Ausprägung von Kinetose, konnte in \cite{vogel motion sickness}\cite{Vogel.1982} außerdem gezeigt werden, dass starke, periodische Längsbeschleunigungen, die vor allem beim Bremsen auftreten, ebenfalls das Auftreten der Reisekrankheit begünstigen. 

Ein bekanntes Nebenphänomen der Reisekrankheit ist, dass sie überwiegend bei Beifahrern bzw. Passagieren, die nicht der Fahrer sind, auftreten \Cite{motion sickness reason}\Cite{Reason.1975}. In \cite{sivak motion sickness} und \cite{rolnick}\cite{Rolnick.1991} wurden mehrere Gründe für dieses Phänomen gefunden. Zum einen hat der Fahrer, als die Person, die das Fahrzeug steuert, die Kontrolle über die Fahrzeugbewegungen und damit auch die Richtung, in die sich die Fahrzeuginsassen bewegen. Dies ist bei den Beifahrern nicht der Fall und die fehlende Kontrolle führt zu einer größeren Anfälligkeit für die Reisekrankheit. Außerdem kann der Fahrer durch die ihm gegebene Kontrolle über das Fahrzeug Bewegungen besser antizipieren. In \cite{sivak motion sickness} wird außerdem der bereits erwähnte visuell-vestibuläre Sensorkonflikt als Grund angeführt, da Beifahrer die Möglichkeit haben sich fahrfremden Tätigkeiten zu widmen, wodurch der Sensorkonflikt verstärkt wird. Bei einer Umfrage zum Thema fahrfremder Tätigkeiten während der Fahrt mit einem vollständig selbst fahrenden Auto in \cite{sivak motion sickness} gab ein Großteil der über 3000 Befragten an, während der Fahrt andere Tätigkeiten als das Beobachten der Straße auszuüben. Gleichzeitig gaben zwischen 4 und 14 Prozent der Erwachsenen an, dass sie in vollständig selbst fahrenden Autos wahrscheinlich oft bis immer ein gewisses Maß an Kinetose verspüren würden. Die Ausprägung moderater bis schwerer Symptome tritt nach eigener Einschätzung bei zwischen 4 und 17 Prozent der Befragten auf. Diese Studie verdeutlicht die besondere Bedeutung, die der Reisekrankheit bei der Entwicklung autonomer Fahrzeuge zukommt. 

Kinetose besitzt aus zwei Gründen eine ausdrückliche Relevanz für die Komfortwahrnehmung beim Autonomen Fahren. Neben dem Einfluss von Beschleunigungen und Rucken, den diese bereits bei manueller, selbst gesteuerter Fahrt auf das Komfortempfinden haben, kommt nun noch dazu, dass auch das Risiko an Kinetose zu erkranken, maßgeblich vom Beschleunigungsverhalten abhängt. Des Weiteren tritt die Reisekrankheit überwiegend bei Beifahrern bzw. Passagieren auf, wodurch bei der Fahrt mit einem autonomen Fahrzeug erhöhtes Kinetoserisiko besteht, wenn sich der Fahrer nun anstelle der Fahraufgabe auf fahrfremde Tätigkeiten wie Zeitunglesen oder die Arbeit am Laptop oder Smartphone konzentrieren kann und dadurch immer mehr selbst zum Beifahrer wird. Dementsprechend besteht die Gefahr, dass Kinetose bei autonomen Fahrten vermehrt auftreten könnte, wodurch der Komfort drastisch reduziert wird. Es ist daher plausibel, dass den Ursachen von Kinetose bei der Entwicklung autonomer Fahrzeuge und der Planung der Fahrzeugbewegung durch entsprechende Maßnahmen entgegengewirkt werden sollte. 

Nachdem der Einfluss unterschiedlicher Faktoren auf den Fahrkomfort - insbesondere unter Berücksichtigung von Kinetose - analysiert wurde, lässt sich feststellen, dass im Beschleunigungsverhalten des Fahrzeugs - sowohl in longitudinaler als auch in lateraler Richtung - die wichtigsten Merkmale für die Komfortbetrachtung bei der Trajektorienplanung enthalten sind. Aus diesem Grund werden in dieser Arbeit neben der Reisezeit ausschließlich die Kriterien Längs- und Querbeschleunigung sowie Längs- und Querruck in verschiedenen Kombinationen als Gütekriterien und zur Beurteilung des Fahrkomforts verwendet. 

\section{Komfortgrenzen für Fahrzeugbeschleunigung und -ruck}
Für die Bewertung der Komfortkriterien hinsichtlich des Fahrkomforts ist es notwendig möglichst genaue Grenzwerte für diese anzugeben. Nachfolgend wird der Frage nachgegangen, inwiefern sich Grenzen der verwendeten Komfortkriterien Fahrzeugruck und -beschleunigung quantitativ bestimmen lassen. Die Grenzen werden anhand von Literaturwerten herausgearbeitet und getrennt nach Längs- und Querdynamik betrachtet.

\subsection{Grenzwerte für die Längsdynamik} 
Für die Begrenzung der Längsdynamik des Fahrzeugs soll zunächst ein Blick auf bestehende \acrshort{FAS} geworfen werden, die den Fahrer in der Längsführung des Fahrzeugs unterstützen. Die Systeme \acrshort{ACC} sowie \acrshort{FSRA} stellen dafür praxiserprobte Systeme dar, die als Orientierung verwendet werden können. Das \acrshort{FSRA} ist eine Systemerweiterung des Standard \acrshort{ACC}, dessen Funktionalität zusätzlich zum mittleren und hohen Geschwindigkeitbereich auch den niedrigen Geschwindigkeitsbereich ($v < \valunit{5}{m/s}$) abdeckt \cite{Winner handbuch FAS}\cite{Winner.2009}. In den ISO-Normen ISO 15622 \cite{iso15622} und ISO 22179 \cite{iso22179} werden die Funktionsgrenzen der beiden Systeme spezifiziert, welche auch in \cite{Winner handbuch FAS}\cite{Winner.2009} nachgelesen werden können. Geht man davon aus, dass der Betrieb autonomer Fahrzeuge im gesamten möglichen Geschwindigkeitsbereich vom Stillstand bis hin zu hohen Geschwindigkeiten, welche auf Autobahnen vorherrschen, möglich sein soll, so dienen die Funktionsgrenzen des \acrshort{FSRA} als erste Anhaltspunkte für autonome Fahrzeuge. \\
Beim \acrshort{FSRA} wird zwischen den einzelnen Geschwindigkeitsbereichen unterschieden. Bei hohen Geschwindigkeiten von $v>\valunit{20}{m/s}$ betragen die Grenzwerte für die maximal zur Verfügung stehende Beschleunigung bzw. Verzögerung $\amax = \valunit{2}{m/s^2}$ bzw. $\amin = \valunit{-3,5}{m/s^2}$ \cite{Winner handbuch FAS}. Im Bereich niedriger Geschwindigkeiten von $v < \valunit{5}{m/s}$ betragen die Funktionsgrenzen $\amax = \valunit{4}{m/s^2}$ bzw. $\amin = \valunit{-5}{m/s^2}$ \cite{Winner handbuch FAS}. Im dazwischen liegenden Geschwindigkeitsbereich dürfen die Höchstwerte für die Beschleunigung und Verzögerung abhängig von der gefahrenen Geschwindigkeit innerhalb der angegebenen Werte für hohe und niedrige Geschwindigkeiten liegen \cite{Winner handbuch FAS}\cite{Winner.2009}.\\
Der Grenzwert für den zulässigen Fahrzeugruck wird für hohe Geschwindigkeiten mit $\jmax = \valunit{2,5}{m/s^3}$ und für niedrige Geschwindigkeite mit $\jmax = \valunit{5}{m/s^3}$ angegeben, wobei dieser Wert sowohl für positive als auch negative Beschleunigungen gilt \cite{Winner handbuch FAS}\cite{Winner.2009}. Auch hier darf der maximale Fahrzeugruck bei mittleren Geschwindigkeiten je nach tatsächlicher Geschwindigkeit innerhalb dieser Grenzen liegen. \\
Somit sind erste Werte für Funktionsgrenzen in Längsrichtung abgesteckt, wobei beachtet werden muss, dass diese Werte als absolute, zulässige Maximalwerte zu verstehen sind, bei denen der Fahrkomfort noch nicht berücksichtigt wurde \cite{Festner.2019}. Es kann davon ausgegangen werden, dass die Komfortgrenzen für \acrshort{FAS} enger gefasst werden müssen als die reinen Funktionsgrenzen, weshalb die Grenzen unter Berücksichtigung des Komforts nun weiter eingeschränkt werden sollen.

In \cite{liu und wu} wird ein Modell zur Folgefahrt basierend auf dem Fahrer- bzw. Passagierkomfort vorgestellt. Als Schwellwert der Längsbeschleunigung für die Unterscheidung zwischen einer komfortablen und einer unkomfortablen Fahrt wird dabei der Wert $a_c = \valunit{2}{m/s^2}$ angegeben. In \cite{radke} wurde die mittlere Längsbeschleunigung eines ``undynamischen'' und damit als komfortorientiert empfundenen Fahrstils auf Landstraßen experimentell als \valunit{1}{m/s^2} ermittelt. 
\textbf{TODO: Längsverzögerung}

Für den Längsruck gilt laut \cite{Fun-to-Drive by Feedback} typischerweise für die Entwicklung vieler Anwendungen wie den Entwurf von Zugstrecken oder Personenaufzügen ein Grenzwert von $\jmax = \pm\valunit{2}{m/s^3}$ zur Wahrung des Passagierkomforts. Neben den absoluten Werten des Rucks konnte Hiroaki in einer Studie für das Railway Technical Research Institute in Japan zudem einen signifikanten Zusammenhang zwischen dem Ruck und der Beschleunigung bezüglich der Nutzerakzeptanz feststellen \cite{hiroaki}. Mit zunehmender Beschleunigung und größer werdendem Ruck sinkt die Akzeptanz eines Fahrmanövers.  


Die Ergebnisse in diesem Kapitel machen die Schwierigkeit bei der objektiven Bestimmung von Grenzwerten für den Fahrkomfort deutlich. Die Komfortwahrnehmung wird von einer Reihe unterschiedlicher Faktoren beeinflusst. Erst deren Gesamtwirken lässt sich je nach Situation und Fahrmanöver bezüglich des Fahrkomforts bewerten. Dazu kommt, dass das Komfortempfinden auch von subjektiven Eindrücken wie dem persönlichen Sicherheitsempfinden oder eigenen Fahrgewohnheiten geprägt wird. Allgemeingültige Werte, die als ``harte'' Komfortgrenzen verstanden werden können, können deshalb ohne zusätzliche Annahmen und Vereinfachungen nicht getroffen werden. Die in diesem Kapitel erarbeiten Grenzwerte für die komfortrelevanten Größen Fahrzeugruck und -beschleunigung in longitudinaler und lateraler Richtung dienen daher eher als Orientierung für die qualitative Bewertung des Fahrkomforts.

\chapter{Fahrzeugmodellierung}\label{cha:Modellbildung}
\chapter{Optimierung - Mathematische Grundlagen und Methoden}\label{cha:Optimierung}
In diesem Kapitel werden die notwendigen mathematischen Grundlagen hergeleitet und dargelegt, die zur Formulierung eines Optimierungsproblems und zur Lösung mithilfe der Variationsrechnung notwendig sind. Dazu wird zunächst der Unterschied zwischen \textit{statischer} und \textit{dynamischer} Optimierung erläutert und anschließend mit der Variationsrechnung eine Lösungsmethode eingeführt, mit der sich ein dynamisches Optimierungsproblem in ein System nichtlinearer \gls{DGL} 1. Ordnung überführen lässt, dessen Lösung entweder analytisch oder numerisch bestimmt werden kann. Danach werden zusätzliche Methoden erläutert, mit deren Hilfe  sich ein komplexes Optimierungsproblem in mehrere miteinander verknüpfte Optimierungsaufgaben unterteilen lässt, deren Lösungen vergleichsweise einfach bestimmt werden können. Abschließend werden einige Lösungsverfahren zur numerischen Lösung dynamischer Optimierungsprobleme vorgestellt und diskutiert. 


\section{Statische Optimierung}\label{sec:statischeOpt}
Die allgemeine Standardformulierung eines statischen Optimierungsproblems lautet \cite{KnutGraichen.2012}:
\begin{align*}
	\min_{\ve{x}\in\mathbb{R}^n} \quad &\fofx \\
	\textrm{s.t.} \quad&\ve{g}(\ve{x}) = 0\\
	&\ve{h}(\ve{x}) \leq 0
\end{align*}
Die Funktion \fofx\,wird dabei als Kosten- oder Gütefunktion bezeichnet und bezüglich der Optimierungsvariablen $\ve{x}$ minimiert. Bei der statischen Optimierung sind die Variablen $\ve{x}$ Elemente des Euklidischen Raums \cite{KnutGraichen.2012}. Für die \gls{GNB} $\ve{g}(\ve{x})$ und die \gls{UNB} $\ve{h}(\ve{x})$ gilt $\ve{g}\in\mathbb{R}^p$ bzw. $\ve{h}\in\mathbb{R}^q$, wobei $p<n$ sein muss, da sich die Optimierungsvariablen $\ve{x}$ ansonsten bei $p=n$ unabhängigen Gleichungen direkt aus $\ve{g}(\ve{x})$ bestimmen lassen \cite{Papageorgiou.2012}. Für die Anzahl der \gls{UNB} hingegen gibt es keine maximale Anzahl \cite{Papageorgiou.2012}.


\section{Dynamische Optimierung}\label{sec:dynamischeOpt}
Im Unterschied zur statischen Optimierung sind die Optimierungsvariablen bei der dynamischen Optimierung selbst Funktionen einer unabhängigen Variable, welche in den meisten Fällen der Zeit $t$ entspricht \cite{KnutGraichen.2012}. Das bedeutet, es werden nun die optimalen Zeitverläufe \xoft\,der Optimierungsvariablen gesucht. Aufgrund dessen wird die Funktion \fofx\,zum Kosten- oder Gütefunktional \J, welches bei der dynamischen Optimierung minimiert wird. Handelt es sich bei den gesuchten Zeitverläufen der Optimierungsvariablen \xoft\,um die Zustände eines dynamischen Systems, so liefert die Lösung des Problems die optimalen Zustandstrajektorien, weshalb die Formulierung eines dynamischen Optimierungsproblems einen geeigneten Ansatz zur Trajektorienplanung dynamischer Systeme darstellt. Werden zusätzlich oder anstatt der Zustandsverläufe die Eingangsgröße des dynamischen Systems als Optimierungsvariable gewählt, so wird auch von einem \textit{Optimalsteuerungsproblem} gesprochen, da die Lösung die optimale Eingangs- bzw. Steuerungstrajektorie liefert \cite{KnutGraichen.2012}.
Die allgemeine Formulierung eines dynamischen Optimierungsproblems lautet \cite{KnutGraichen.2012}:
\begin{align}
\min_{\xoft,\uoft} \quad &J(\xoft,\uoft,t) = \Vofxoft + \int_{t_0}^{t_f}l(\xoft,\uoft,t)\dtint{t} \label{eq:J_dyn}\\
\textrm{s.t.} \quad& \dx = \ve{f}(\ve{x},\ve{u},t)\,,\qquad \ve{x}(t_0) = \ve{x}_0 \label{eq:system_anfang}\\
&\ve{g}(\ve{x}(t_f),t_f) = 0 \label{eq:GNB}\\
&\ve{h}(\ve{x},\ve{u}) \leq 0 \label{eq:UNB}
\end{align}
Wie bereits bei der statischen Optimierung stellen die Gleichungen \ref{eq:GNB} und \ref{eq:UNB} \gls{GNB} und \gls{UNB} dar. Zusätzlich beschreibt Gleichung \ref{eq:system_anfang} die Systemdynamik mit der Eingangsgröße \uoft\,und die Anfangszustände $\ve{x}_0$ des Systems, die als \gls{RB} fungieren. Das Gütefunktional \J\,wird in der dargestellten Form auch als \textit{Bolza-Form} des Gütefunktionals bezeichnet und setzt sich aus zwei Teilen zusammen: Der Integralanteil beschreibt die von der Zeit abhängenden laufenden Kosten und heißt \textit{Lagrange-Form}. Der vor dem Integralanteil stehende Term \Vofxoft\,gibt die Bewertung des Endzustands (und der Endzeit), also die Endkosten, an. Dieser wird \textit{Mayer-Form} genannt \cite{KnutGraichen.2012}. Für die Lagrange- und Bolza-Form gilt, dass sie sich immer in die Mayer-Form überführen lassen \cite{KnutGraichen.2012} (siehe auch \cite{Gerdts.2010}). Der Endzeitpunkt $t_f$ kann allgemein festgelegt sein oder auch frei. Ist er frei, so muss $t_f$ als zusätzliche Optimierungsvariable bei der Lösung des Optimierungsproblems berücksichtigt werden. 

\subsection{Variationsrechnung}
Die Variationsrechnung bietet einen Ansatz, mit dem dynamische Optimierungsprobleme, wie im vorherigen Absatz vorgestellt, gelöst werden können. Dazu werden ausgehend von den optimalen Trajektorien \opt{\ve{x}} und \opt{\ve{u}} und dem optimalen Endzeitpunkt \opt{t_f} (sofern $t_f$ Teil der Optimierung ist) Variationen zugelassen, sodass gilt \cite{KnutGraichen.2012}:
\begin{align}
\xoft &= \opt{\ve{x}}(t) + \epsilon\variation{x}(t) \label{eq:x_var}\\
\dxoft &= \opt{\dx}(t) + \epsilon\variation{\dot{x}}(t) \label{eq:dx_var}\\
\uoft &= \opt{\ve{u}}(t) + \epsilon\variation{u}(t) \label{eq:u_var}\\
t_f &= \opt{t_f} + \epsilon\delta_{t_f}\label{eq:t_var}
\end{align}
Die $\ve{\delta}$-Variablen bezeichnen dabei die Variationen der einzelnen Größen und $\epsilon$ ist der Parameter, mit dem die Variationen Einfluss auf die Trajektorien nehmen, wobei für $\epsilon=0$ offensichtlich $\xoft=\opt{\ve{x}}(t)$, $\dxoft = \opt{\dx}$ und $\uoft=\opt{\ve{u}}(t)$ gilt. In Abbilung \ref{fig:Variation} ist ein solcher Verlauf einer möglichen optimalen Trajektorie $\opt{x}(t)$ und einer zulässigen Variation skizziert. Außerdem sind die Variation des Endzeitpunkts und des Endzustands $\ve{x}(t_f) = \opt{\ve{x}}(t_f)+\variation{x}(t_f)+\opt{\dx}(\opt{t_f})\delta_{t_f}$ (falls, $x(t_f)$ frei ist) dargestellt.
\begin{figure}[h]
\centering
\begin{tikzpicture}[domain=0:4, samples=200]
\draw[->] (0,0) -- (5,0) node[right] {$t$};
\draw[->] (0,0) -- (0,4) node[above] {$x$};
\draw[color=black,domain=0:4,-*]    plot (\x,{sqrt(\x)});
\draw[color=red,domain=0:4.5,-*]   plot (\x,{sqrt(\x)+0.5*sin(3*\x r)});
\draw [dashed] (3.92,0) node[below] {$\opt{t_f}$} -- (3.92,3);
\draw [dashed] (4.43,0) node[below] {$t_f$} -- (4.43,3);
\draw [stealth-stealth] (3.92,0.7) -- (4.43,0.7) node[midway,above] {$\delta_{t_f}$};
\draw [stealth-stealth] (4.7,1.9) -- (4.7,2.51) node[midway,right] {$\variation{x}(t_f)+\opt{\dx}(\opt{t_f})\delta_{t_f}$};
\draw [-stealth] (2.7,0.7) node[below] {$\opt{\ve{x}} + \epsilon\variation{x}(t)$} -- (3.3,1.4) ;
\draw [-stealth] (1.3,2) node[above] {\opt{\ve{x}}} -- (1.5,1.3) ;
\end{tikzpicture}
\caption{Schematische Darstellung der optimalen Trajektorie $\opt{\ve{x}}(t)$ und eine mögliche, zulässige Variation \xoft.}
\label{fig:Variation}
\end{figure}

Allgemein wird für die Herleitung der Lösung des Variationsproblems ein Gütefunktional der Form
\begin{equation}
	J(\xoft,\dxoft,t) = \int_{t_0}^{t_f}\Phi(\xoft,\dxoft,t)\dtint{t} \label{eq:J_var}
\end{equation}
verwendet \cite{KnutGraichen.2012}, welches für die Optimierung dynamischer Systeme wie folgt definiert werden kann \cite{KnutGraichen.2012}:
\begin{equation}
\bar{J}(\xoft,\dxoft,\uoft,t) = \Vofxoft + \int_{t_0}^{t_f}l(\xoft,\uoft,t) + \ve{\lambda}^T(\ve{f}(\ve{x},\ve{u},t) - \dx)\dtint{t} \label{eq:J_var_sys}
\end{equation}
Im Vergleich zu \ref{eq:J_dyn} wurde das Gütefunktional um den Term $\ve{\lambda}^T(\ve{f}(\ve{x},\ve{u},t) - \dx)$ erweitert, wobei dieser für $\dx = \ve{\lambda}^T(\ve{f}(\ve{x},\ve{u},t)$ zu null wird. Die Variablen $\ve{\lambda}$ stellen dabei die sogenannten adjungierten Zustände dar \cite{KnutGraichen.2012}. Werden die Variablen in \ref{eq:J_var} nun durch die Gleichungen \ref{eq:x_var}-\ref{eq:t_var} ersetzt, hängt das Gütefunktional nur noch von $\epsilon$ ab. Da die Trajektorien \xoft, \dxoft und \uoft wie bereits gezeigt nur für $\epsilon=0$ den optimalen Verläufen entsprechen können, lautet die notwendige Bedingung für das Verschwinden der Variation des Gütefunktionals $\delta_J$ und damit für ein Minimum $\delta_J = \frac{\textrm{d}J(\opt{\ve{x}}+\epsilon\variation{x})}{\dtint{\epsilon}}_{|\epsilon=0}=0$ bzw. $\delta_{\bar{J}} = \frac{\textrm{d}\bar{J}(\opt{\ve{x}}+\epsilon\variation{x})}{\dtint{\epsilon}}_{|\epsilon=0}=0.$
Aus dieser Bedingung lässen sich die notwendigen Optimalitätbedingungen für eine optimale Lösung herleiten. Auf die vollständige Herleitung der Gleichungen wird an dieser Stelle verzichtet und auf weiterführende Literatur verwiesen \cite{KnutGraichen.2012,Papageorgiou.2012,Gerdts.2010}.


% =================================================================================
% Anhang
% =================================================================================
\appendix % Damit wird der Anhang begonnen. Die Kapitel werden ab jetzt mit Buchstaben nummeriert


% =================================================================================
% Literaturverzeichnis
% =================================================================================
\cleardoublepage        % Auf eine leere Seite einfügen
\phantomsection         % Für Aufnahme ins Inhaltsverzeichnis
\addcontentsline{toc}{chapter}{\bibname}  % In Inhaltsverzeichnis von
                                          % Dokument und pdf aufnehmen
\printbibliography 
% =================================================================================

	
\end{document}
